\documentclass[12pt, a4paper]{article}
\usepackage{amsmath}
\usepackage{amsfonts}
\usepackage{amsthm}
\usepackage{mathtools}
\newtheorem{theorem}{Theorem}[section]
\newtheorem{definition}{Definition}[section]
\numberwithin{equation}{section}
\usepackage{pgfplots}
\pgfplotsset{width=10cm,compat=1.9}
\graphicspath{ {img/} }
\DeclareGraphicsExtensions{.png}

\title{Differential Geometry}
\author{Kristian Wichmann}

\begin{document}
\maketitle

\section{Regular curves in Euclidean spaces}
Let $V$ be a finite-dimensional vector space over $\mathbb{R}$. Then a $C^k$ \textit{regular curve} in $V$ is a mapping:
\begin{equation}
x: I\rightarrow V
\end{equation}
Here $I$ is an interval, and we require that $x$ is $C^k$ and that $x'(t)\neq 0$ for any $t\in I$ (this last requirement is what makes the curve regular).

Given a $C^k$ regular curve $x: I\rightarrow V$. Then if $\varphi$ is a $C^k$ bijection from the interval $J$ to $I$ having $\varphi'>0$, we consider $x$ equivalent to the regular curve $y: J\rightarrow V, y=x\circ\varphi$.

We now assume $V$ has an inner product $(\cdot,\cdot)$ and an induced norm $||v||=\sqrt{(v,v)}$. Then for a regular curve $x$ we define an \text{arc length function} $s(t)$ by:
\begin{equation}
\frac{ds}{dt}=||x'(t)||
\end{equation}
Or equivalently:
\begin{equation}
s(t)=\int||x'(t)||\ dt
\end{equation}
This means that two arc length functions differ only by a constant. Since $x'$ is never zero, $||x'(t)||>0$, and so $s(t)$ is strictly increasing. As a integral of a $C^{k-1}$ function it is itself $C^k$.

We now definte the function $X(s(t))=x(t)$. Being a composition of two $C^k$ functions, so is $X$. So we can always parametrize a regular curve by its arc length. The derivative of $X$ is found by the chain rule:
\begin{equation}
\frac{dX}{ds}=\frac{dx}{dt}\frac{dt}{ds}=x'(t(s))\frac{1}{ds/dt}=\frac{x'(t(s))}{||x'(t(s))||}
\label{x_prime_s}
\end{equation}
Here we've used, that since there is a bijection between $I$ and $s(t)$ we can regard $t$ as a function of $s$. Taking the norm, we get $||X'(s)||=1$. Squaring this gives $||X'(s)||^2=(X'(s),X'(s))=1$. Differentiate with respect to $s$ to get:
\begin{equation}
2(X''(s),X'(s))=0
\end{equation}
So the second derivate is orthogonal to the tangent $X'(s)$.

\subsection{Example: Circular curve}
Let $x_0\in V$ and let $r_1,r_2\in V$ be perpendicular vectors, each of length $R$. Then we can make a circular curve centered in $x_0$, radius $R$, and in the plane spanned by $r_1$ and $r_2$ by:
\begin{equation}
x(s)=x_0+r_1\cos(s/R)+r_2\sin(s/R)
\end{equation}
The tangent, i.e. the first derivative is:
\begin{equation}
x'(s)=\frac{1}{R}[-r_1\sin(s/R)+r_2\cos(s/R)]
\end{equation}
The second derivative is:
\begin{equation}
x''(s)=-\frac{1}{R^2}[r_1\sin(s/R)+r_2\cos(s/R)]=-\frac{1}{R^2}(x(s)-x_0)
\end{equation}
This is pointed at the center of the circle and has a magnitude of $||x''(s)||=1/R$.

\subsection{Curvature of regular Euclidean curves}
Inspired by the circular curve example, we make the following definition:
\begin{definition}
Given a regular $C^2$ curve in $V$ parametrized by arc length: $x(s)$ and a point on the curve where $x''(s)\neq 0$.

Then $n=x''(s)/||x''(s)||$ is called the principal normal to the curve at $x(s)$.

$1/||x''(s)||$ is known as the radius of curvature at $x(s)$. The circle with center at $x(s)+x''(s)/||x''(s)||$, radius equal to the radius of curvature, in the plane spanned by $x'(s)$ and $n$ is known as the osculating circle.

$\kappa=||x''(s)||$ is called the curvature of the curve at $x(s)$, even if equal to zero. So when $\kappa\neq 0$ we have $x''(s)=n\kappa$
\end{definition}

What if out curve $x(t)$ is not parametrized by arc length $s$? Then we can still calculate curvature by using the $X$ function: $x(t)=X(s(t))$. The first dervative of $x$ can be found using the chain rule:
\begin{equation}
x'(t)=\frac{dX}{dt}=\frac{dX}{ds}\frac{ds}{dt}=X'(s)\frac{ds}{dt}
\end{equation}
And the second derivative by the multiplication rule:
\begin{equation}
x''(t)=X''(s)\left(\frac{ds}{dt}\right)^2+X'(s)\frac{d^2s}{dt^2}
\end{equation}
Remember that $ds/dt=||x'(t)||$:
\begin{equation}
x''(t)=X''(s)||x'(t)||^2+X'(s)\frac{d}{dt}||x'(t)||
\end{equation}
Consider the last derivative:
\begin{equation}
\frac{d}{dt}||x'(t)||=\frac{d}{dt}\sqrt{(x'(t),x'(t))}=\frac{1}{2\sqrt{(x'(t),x'(t))}}2(x'(t),x'')=\frac{(x'(t),x''(t))}{||x'(t)||}
\end{equation}
Rearranging terms and using equation \ref{x_prime_s} we get:
\begin{equation}
X''(s)||x'(t)||^2=x''(t)-\frac{(x'(t),x''(t))}{||x'(t)||^2}x'(t)
\end{equation}
Finally we can find the second derivative of $X$:
\begin{align}
X''(s)&=\frac{x''(t)}{||x'(t)||^2}-\frac{(x'(t),x''(t))}{||x'(t)||^4}x'(t)\\
&=\frac{1}{||x'(t)||^2}\left(x''(t)-\frac{(x'(t),x''(t))}{||x'(t)||^2}x'(t)\right)
\end{align}
This means that the curvature is:
\begin{equation}
\kappa=||X''(s)||=\frac{1}{||x'(t)||^2}\left\Vert x''(t)-\frac{(x'(t),x''(t))}{||x'(t)||^2}x'(t)\right\Vert
\end{equation}

\subsection{Regular curves in $\mathbb{R}^2$}
Let $x(t)$ be a regular curve in $\mathbb{R}^2$. Then we may attach a sign to the curvature $\kappa$. Let $x(s)$ be a parametrization of the curve by arc length. Then if $||x''(s)||\neq 0$, we know that $x'(s)$ and $x''(s)$ are perpendicular. Because we're in two dimensions, $x''(s)$ is either rotated 90 degrees in positive of negative direction (conter-clockwise or clockwise). This determines the sign of the curvature.

Now consider a regular curve $x(s)$ in $\mathbb{R}^2$ that is also closed and simple, i.e. one that is periodic, and has no self-intersections apart from this. If the arc length is $L$ we will still let the parametrization be defined for all $s\in\mathbb{R}$, so that:
\begin{equation}
x(s)=x(s')\Leftrightarrow s'-s=nL, n\in\mathbb{Z}
\end{equation}
Since $||x'(s)||=1$ we can parametrize:
\begin{equation}
x'(s)=(\cos(\theta(s)),\sin(\theta(s)), 0\le s\le L
\end{equation}
Now differentiate to find $x''(s)$:
\begin{equation}
x''(s)=(-\sin(\theta(s)),\cos(\theta(s)))\frac{d\theta}{ds}=\widehat{x'(s)}\frac{d\theta}{ds}
\end{equation}
Which means that the signed curvature is $\kappa(s)=d\theta/ds$. So we can integrate:
\begin{equation}
\int_0^L\kappa(s)\ ds=\theta(L)-\theta(0)
\label{2d_curvature_integral}
\end{equation}
This means that the integral must be a multiple of $2\pi$.

\begin{theorem}
The integral \ref{2d_curvature_integral} is equal to $\pm \pi$, with the positive sign if the curve lies entirely to the left of some tangent line and the overall motion is counter-clockwise.
\end{theorem}
\begin{proof}
Without loss of generality we can assume that $x(0)$ is the point with the lowest second coordinate value. This implies that $x'(0)=(1,0)$, i.e. in the direction of the first axis because of the counter-clockwise motion.

Define the relative cord direction:
\begin{equation}
x(t,s)=\frac{x(t)-x(s)}{||x(t)-x(s)||}, 0\le s<t\le L
\end{equation}
This is well-defined since the curve is closed, so the denominator never becomes zero. When $s=t$ we set $x(s,s)=x'(s)$, which makes the function continuous on the triangle $T=\{(t,s)\in\mathbb{R}^2|0\le s\le t\le T\}$.

$T$ is simply connected and by construction $||x(t,s)||=1$, so there exists a continuous function $\varphi: T\rightarrow\mathbb{R}^2$ so that:
\begin{equation}
x(t,s)=(\cos(\varphi(t,s)),\sin(\varphi(t,s))),\quad\varphi(s,s)=\theta(s)
\end{equation}
We wish to evaluate:
\begin{equation}
\theta(L)-\theta(0)=\varphi(L,L)-\varphi(0,0)
\end{equation}
Because of the chosen geometry, we have $x(0, 0)=x'(0)=(1,0)$. So $\varphi(0,0)$ must be a multiple of $2\pi$. We choose 0. So we only need to consider the value of $\varphi(L,L)$.

To do this, start by noticing, that since we chose $\varphi(0,0)=0$, we can never have $\varphi(L,L)>2\pi$, because of the location of $x(0)$. Now consider $\varphi(L,t)$ in the limit $t\rightarrow L$. Again, because of the geometry, we must have $\varphi(L,L)\le 2\pi$. This leaves $2\pi$ as the only option.
\end{proof}

\end{document}