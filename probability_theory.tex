\documentclass[12pt, a4paper]{article}
\usepackage{amsmath}
\usepackage{amsfonts}
\usepackage{amsthm}
\usepackage{mathtools}
\newtheorem{theorem}{Theorem}[section]
\newtheorem{definition}{Definition}[section]

\title{Probability theory}
\author{Kristian Wichmann}

\begin{document}

\maketitle

This is an overview of probability theory expressed in the language of measure theory.

\section{Probability spaces}
\begin{definition}
A probability space is a measure space $(\Omega, \mathcal{F}, P)$ for which $P(\Omega)=1$. The elements of $\Omega$ are called outcomes, the elements of $\mathcal{F}$ events and $P(F)$ is the probability of the event $F\in\mathcal{F}$.
\end{definition}

\section{Random variables}
\begin{definition}
An $E$-valued random variable $X$ on a probability space $(\Omega, \mathcal{F}, P)$ is a measurable function $X: \Omega\rightarrow E$. Here $(E,\mathcal{E})$ is a measurable space. Often this measurable space is $(\mathbb{R},\mathbb{B})$, so that $X: \Omega\rightarrow\mathbb{R}$. Such a real-valued random variable is usually simply denoted a random variable for brevity.
\end{definition}


\subsection{Distribution}
Billedmålet giver pdf

\section{Statistical models}

\section{Likelihood}


\end{document}