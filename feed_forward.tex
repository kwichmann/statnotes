\documentclass[12pt, a4paper]{article}
\usepackage{amsmath}
\usepackage{amsfonts}
\usepackage{amsthm}
\usepackage{mathtools}
\newtheorem{theorem}{Theorem}[section]
\newtheorem{definition}{Definition}[section]
\numberwithin{equation}{section}
\usepackage{pgfplots}
\pgfplotsset{width=10cm,compat=1.9}
\graphicspath{ {img/} }
\DeclareGraphicsExtensions{.png, .jpg}

\title{General feed forward neural networks}
\author{Kristian Wichmann}

\begin{document}
\maketitle

\section{Formalism and nomenclature}
Consider a feed forward neural net with $L+1$ layers, in the sense that layer zero is the input layer and layer $L$ the output layer. Each non-input layer has its own activation function $\sigma_l$.

The size of layer $l$ we will denote $S_l$. So layer $l$ has a $S_l\times S_{l-1}$ matrix of weights $W^{(l)}$ and a bias vector $b^{(l)}$ with dimension $S_l$. So, given an input vector $x$ (with dimension $S_{l-1}$), the pre-activation and the activation of layer $l$ can be expressed as:
\begin{equation}
\label{forward_propagation}
z^{(l)}=W^{(l)} x + b^{(l)},\quad a^{(l)}=\sigma_l(z^{(l)})=\sigma_l(W^{(l)} x + b^{(l)})
\end{equation}

We will consider $N$ data points, each with a feature vector of dimension $S_0$. We group these into an $S_0\times N$ matrix $X$:
\begin{equation}
X=
\begin{pmatrix}
| & \cdots & | \\
x_1 & \cdots & x_N \\
| & \cdots & |
\end{pmatrix}
\end{equation}
Note that this is the transpose of the usual "data frame" structure.

We can now write the pre-activations and activations of the first layer as:
\begin{equation}
Z^{(1)}=W^{(1)} X+b^{(1)},\quad A^{(1)}=\sigma_1(Z^{(1)})=\sigma_1(W^{(1)} X+b^{(1)})
\end{equation}
Here, we've used "$+b^{(1)}$" as a shorthand for adding the vector $b^{(1)}$ to every column. We could write this as "$+b^{(1)} J^t_{N}$" if we wanted to be accurate. ($J_N$ is a column vector of $N$ ones).

Similarly, we may generally write the pre-activations and activations of layer $l$ as:
\begin{equation}
Z^{(l)}=W^{(l)} A^{(l-1)}+b^{(l)},\quad A^{(l)}=\sigma_l(Z^{(l)})=\sigma_l(W^{(l)} A^{(l-1)}+b^{(l)})
\end{equation}

Finally, we have an error function $J$ which measures the distance to some target data $T\in\mathbb{R}^{S_L\times N}$:
\begin{equation}
T=
\begin{pmatrix}
| & \cdots & | \\
t_1 & \cdots & t_N \\
| & \cdots & |
\end{pmatrix}
\end{equation}
We will assume the error function is of the form $J=J(T,A^{(L)})$, taking on real values. I.e. it only depends on the targets and the activations of the output layer.

\section{Backpropagation}
Forward propagation through the network is described by equation \ref{forward_propagation}. The procedure is assumed to be done before we look at how to determine partial derivatives of $J$ through backpropagation.

\subsection{Output layer}
The derivatives with respect to the output layer weights and biases will take a rather abstract form in this general formalism:
\begin{equation}
\label{output_grad}
\frac{\partial J}{\partial W^{(L)}_{ij}}=\sum_{k=1}^{S_L}\sum_{n=1}^N\frac{\partial J}{\partial A^{(L)}_{kn}}\frac{\partial A^{(L)}_{kn}}{\partial W^{(L)}_{ij}},\quad\frac{\partial J}{\partial b^{(L)}_i}=\sum_{k=1}^{S_L}\sum_{n=1}^N\frac{\partial J}{\partial A^{(L)}_{kn}}\frac{\partial A^{(L)}_{kn}}{\partial b^{(L)}_i}
\end{equation}
We may find the derivatives of $A^{(L)}$ with respect to the weights and biases:
\begin{equation}
\label{output_a_grad}
\frac{\partial A^{(L)}_{kn}}{\partial W^{(L)}_{ij}}=\frac{\partial\sigma_L}{\partial Z^{(L)}_{kn}}\frac{\partial Z^{(L)}_{kn}}{\partial W^{(L)}_{ij}}
\end{equation}
Finally, we may calculate the derivatives of $Z^{(L)}$ with respect to the weights:
\begin{equation}
\frac{\partial Z^{(L)}_{kn}}{\partial W^{(L)}_{ij}}=\frac{\partial}{\partial W^{(L)}_{ij}}(W^{(L)}A^{(L-1)}+b^{(L)})_{kn}=\frac{\partial}{\partial W^{(L)}_{ij}}\left(\sum_{l=1}^{S_{L-1}}W^{(L)}_{kl}A^{(L-1)}_{ln}+b^{(L)}_k\right)
\end{equation}
Differentiating $W^{(L)}$ with respect to $W^{(L)}$ yields two Kronecker deltas:
\begin{equation}
\label{output_z_grad}
\frac{\partial Z^{(L)}_{kn}}{\partial W^{(L)}_{ij}}=\sum_{l=1}^{S_{L-1}}\delta_{ik}\delta_{jl}A^{(L-1)}_{ln}=\delta_{ik} A^{(L-1)}_{jn}
\end{equation}
Now, we may insert equations \ref{output_a_grad} and \ref{output_z_grad} into \ref{output_grad}:
\begin{equation}
\frac{\partial J}{\partial W^{(L)}_{ij}}=\sum_{k=1}^{S_L}\sum_{n=1}^N\frac{\partial J}{\partial A^{(L)}_{kn}}\frac{\partial\sigma_L}{\partial Z^{(L)}_{kn}}\delta_{ik} A^{(L-1)}_{jn}=\sum_{n=1}^N\frac{\partial J}{\partial A^{(L)}_{in}}\frac{\partial\sigma_L}{\partial Z^{(L)}_{in}}A^{(L-1)}_{jn}
\end{equation}
We can rewrite this in matrix form by using the Hadamard product between the two derivatives and swapping the indices of $A^{(L-1)}$, turning into a transpose:
\begin{equation}
\label{output_weights_matrix}
\frac{\partial J}{\partial W^{(L)}}=\left[\frac{\partial J}{\partial A^{(L)}}\odot\frac{\partial\sigma_L}{\partial Z^{(L)}}\right]\left(A^{(L-1)}\right)^t
\end{equation}
The procedure for the biases is the same until we get to the $Z^{(L)}$ derivative:
\begin{equation}
\frac{\partial Z^{(L)}_{kn}}{\partial b^{(L)}_i}=\frac{\partial}{\partial b^{(L)}_i}\left(\sum_{l=1}^{S_{L-1}}W^{(L)}_{kl}A^{(L-1)}_{ln}+b^{(L)}_k\right)=\delta_{ik}
\end{equation}
Reinserting all the way back to equation \ref{output_grad} we get:
\begin{equation}
\frac{\partial J}{\partial b^{(L)}_i}=\sum_{k=1}^{S_L}\sum_{n=1}^N\frac{\partial J}{\partial A^{(L)}_{kn}}\frac{\partial\sigma_L}{\partial Z^{(L)}_{kn}}\delta_{ik}=\sum_{n=1}^N\frac{\partial J}{\partial A^{(L)}_{in}}\frac{\partial\sigma_L}{\partial Z^{(L)}_{in}}
\end{equation}
Again, we can write this is matrix notation using the Hadamard product:
\begin{equation}
\label{output_bias_matrix}
\frac{\partial J}{\partial b^{(L)}_i}=\left[\frac{\partial J}{\partial A^{(L)}}\odot\frac{\partial\sigma_L}{\partial Z^{(L)}}\right]J_N^t
\end{equation}

\subsubsection{Example error function: Euclidean distance}
A common form of error function is:
\begin{equation}
J(T,A^{(L)})=\frac{1}{2N}\sum_{m=1}^N||a^{(L)}_m-t_m||^2
\end{equation}
Here, $a^{(L)}_m$ is the $m$'th column of $A^{(L)}$ and the double dashes is the usual Euclidean norm in $S_L$ dimensions. So we may rewrite:
\begin{equation}
J(T,A^{(L)})=\frac{1}{2N}\sum_{m=1}^N\sum_{j=1}^{S_L}(A^{(L)}_{jm}-T_{jm})^2
\end{equation}
Now we can find the derivative:
\begin{equation}
\frac{\partial J}{\partial A^{(L)}_{in}}=\frac{1}{2N}\sum_{m=1}^N\sum_{j=1}^{S_L}\frac{\partial}{\partial A^{(L)}_{in}}(A^{(L)}_{jm}-T_{jm})^2
\end{equation}
Again, we get two Kronecker deltas:
\begin{equation}
\frac{1}{2N}\sum_{m=1}^N\sum_{j=1}^{S_L}2(A^{(L)}_{jm}-T_{jm})\delta_{ij}\delta_{nm}=\frac{1}{N}(A^{(L)}_{in}-T_{in})
\end{equation}
The quantity in parenthesis is the $ij$ element of the error term matrix for the output layer:
\begin{equation}
\Delta^{(L)}=A^{(L)}_{in}-T_{in}
\end{equation}
So we can write the derivative in matrix form:
\begin{equation}
\frac{\partial J}{\partial A^{(L)}}=\Delta^{(L)}
\end{equation}



\end{document}