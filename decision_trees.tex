\documentclass[12pt, a4paper]{article}
\usepackage{amsmath}
\usepackage{amsfonts}
\usepackage{amsthm}
\usepackage{mathtools}
\newtheorem{theorem}{Theorem}[section]
\newtheorem{definition}{Definition}[section]
\numberwithin{equation}{section}
\usepackage{pgfplots}
\pgfplotsset{width=10cm,compat=1.9}
\graphicspath{ {img/} }
\DeclareGraphicsExtensions{.png}

\title{Decision Trees}
\author{Kristian Wichmann}

\begin{document}
\maketitle

\section{Simple binary decision tree}
Imagine we wish to predict whether a person is more likely to die young (below 70 years old) or old based on whether they smoked and/or drank alcohol. Table \ref{table:drink_smoke_age} shows such data for 100 subjects.

\begin{table}
\centering
\label{table:drink_smoke_age}
\begin{tabular}{cccc}
\hline
\multicolumn{1}{|c|}{Drinker} & \multicolumn{1}{c|}{Smoker} & \multicolumn{1}{c|}{High age} & \multicolumn{1}{c|}{Number} \\ \hline
Y                             & Y                           & Y                             & 2                           \\
Y                             & Y                           & N                             & 16                          \\
Y                             & N                           & Y                             & 20                          \\
Y                             & N                           & N                             & 17                          \\
N                             & Y                           & Y                             & 5                           \\
N                             & Y                           & N                             & 9                           \\
N                             & N                           & Y                             & 28                          \\
N                             & N                           & N                             & 3                          
\end{tabular}
\caption{Table of the 100 subjects}
\end{table}

The predictiontion can be expressed in a binary decision tree. The question is how to make such a tree.

\subsection{Maximizing information gain}
Recall that the \textit{entropy} of a probability distribution is defined as follows:
\begin{equation}
H=-\sum_i p_i\log_2 p_i
\end{equation}
\textit{Information} and entropy are complementary quantities: When entropy is lowered, information is gained. So to maximize information gain is to maximize entropy loss.

\subsubsection{Pre-split entropy}
Before splitting the subjects according to being drinkers or smokers, there's two groups:
\begin{itemize}
\item Subjects with high age: $2+20+5+28=55$.
\item Subjects with low age: $16+17+9+3=45$.
\end{itemize}
So the entropy is:
\begin{equation}
H=-\frac{55}{100}\log_2\frac{55}{100}-\frac{45}{100}\log_2\frac{45}{100}
\end{equation}

\end{document}