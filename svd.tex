\documentclass[12pt, a4paper]{article}
\usepackage{amsmath}
\usepackage{amsfonts}
\usepackage{amsthm}
\usepackage{mathtools}
\newtheorem{theorem}{Theorem}[section]
\newtheorem{definition}{Definition}[section]
\numberwithin{equation}{section}
\usepackage{pgfplots}
\pgfplotsset{width=10cm,compat=1.9}
\graphicspath{ {img/} }
\DeclareGraphicsExtensions{.png}

\title{Single value decomposition and pseudo-inverses}
\author{Kristian Wichmann}

\begin{document}
\maketitle

\section{Gramian matrices}
Given a set of vectors $a_1, a_2,\ldots, a_n\in\mathbb{R}^m$, the Gramian matrix is the traditionally matrix of inner products $\langle a_i,a_j\rangle$. If these vectors are collected into a $m\times  n$ matrix $A$, this matrix can be expressed as $A^t A$. Here, we will use the term for any matrix in this form. By starting out with the transpose instead, this means that $AA^t$ is also a Gramian, with dual results.

\begin{theorem}
If $A\in\mathbb{R}^{m\times n}$, then $A^t A$ is symmetric and positive semi-definite. Iff $A$ has rank $m$, $A^t A$ is positive definite.
\end{theorem}
\begin{proof}
$(A^t A)^t=A^t(A^t)^t=A^t A$ shows symmetry. positive semi-definiteness, let $x\in\mathbb{R}^n$. Then:
\begin{equation}
x^t A^t Ax=\langle Ax,Ax\rangle=||Ax||^2
\end{equation}
As a norm, this is greater than or equal to zero. Hence $A^t A$ is positive semi-definite. If $A$ has rank $m$ the map $x\mapsto Ax$ has a trivial kernel by the rank-kernel theorem. Which means only the zero vector is mapped to zero, and hence $A^t A$ is positive definite. If the rank is less than $m$, the kernel is non-trivial and positive definiteness cannot be true.
\end{proof}

\section{Single value decomposition}
Let $A\in\mathbb{R}^{m\times n}$. Since $A^t A$ is symmetric, it is diagonalizable. So there is an orthogonal $n\times n$ matrix $O$ such that $A^t A=ODO^t$, where $D$ is a diagonal matrix of eigenvalues. 

\end{document}