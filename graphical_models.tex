\documentclass[12pt, a4paper]{article}
\usepackage{amsmath}
\usepackage{amsfonts}
\usepackage{amsthm}
\usepackage{mathtools}
\usepackage{tikz}
\usetikzlibrary{bayesnet}
\newtheorem{theorem}{Theorem}[section]
\newtheorem{definition}{Definition}[section]
\numberwithin{equation}{section}
\usepackage{pgfplots}
\pgfplotsset{width=10cm,compat=1.9}
\graphicspath{ {img/} }
\DeclareGraphicsExtensions{.png}

\title{Probabilistic graphical models}
\author{Kristian Wichmann}

\begin{document}
\maketitle

\section{Bayesian networks}
A group of graphical models are \textit{Bayesian networks}, also know as \textit{directed graphical models}.

\subsection{An example}
As an example, consider three random variables $A, B$, and $C$ with a joint distribution $p(a,b,c)$. Then we can use the multiplication rule of probability to rewrite this as:
\begin{equation}
p(a,b,c)=p(c|a,b)p(a,b)
\end{equation}
Using the same idea once more, this can be further rewritten:
\begin{equation}
p(a,b,c)=p(c|a,b)p(a,b)=p(c|a,b)p(b|a)p(a)
\label{fully_connected3}
\end{equation}
We can interpret this as $c$ depending on $a$ and $b$, and $b$ depending on $a$. We will use directed \textit{edges} to denote this dependency as shown in figure \ref{graph:bn3}.

\begin{figure}
\centering
\tikz{ %
        \node[latent] (c) {$c$} ; %
        \node[latent, left=of c, yshift=0.8cm] (a) {$a$} ; %
        \node[latent, left=of c, yshift=-0.8cm] (b) {$b$} ; %
        \edge {a,b} {c} ; \edge {a} {b} ; %
      }
\caption{A Bayesian network with three nodes}
\label{graph:bn3}
\end{figure}

Do note, that we might as we might as well have split the the variables into dependency in another order! This graph is merely one way of viewing the joint distribution - in general a distribution can be represented by many different graphs.

\subsection{General case: Fully connected}
Using the same scheme, we can decompose a joint distribution of $K$ variables and split it into a form similar to equation \ref{fully_connected3} by applying the multiplication rule of probability $K-1$ times:
\begin{equation}
p(x_1,x_2,\cdots,x_K)=p(x_K|x_1,\cdots,x_{K-1})\cdots p(x_2|x_1)p(x_1)
\label{fully_connectedK}
\end{equation}
This gives rise to a \textit{fully connected} graph, as there will be edges between any two nodes. 

\subsection{Less than fully connected}
However, in general we will be more interested in cases where some pairs of nodes will not have an edge between them. We will still require the graph to be \textit{acyclical}, i.e. there must be no loops when following the directed edges.

\begin{figure}
\centering
\tikz{ %
        \node[latent] (x1) {$x_1$} ; %
        \node[latent, right= of x1] (x2) {$x_2$} ; %
        \node[latent, right= of x2] (x3) {$x_3$} ; %
        \node[latent, right= of x1, yshift=-1.5cm] (x4) {$x_4$} ; %
        \node[latent, right= of x4] (x5) {$x_5$} ; %
        \node[latent, right= of x1, yshift=-3cm] (x6) {$x_6$} ; %
        \edge {x1,x2,x3} {x4} ; \edge {x3} {x5} ; 
        \edge {x4,x5} {x6} ; %
      }
\caption{A less than fully connected Bayesian network}
\label{graph:bn_notfull}
\end{figure}

As an example, consider the six-node graph in figure \ref{graph:bn_notfull}. The corresponding decomposition would be:
\begin{equation}
p(x_1, x_2, x_3, x_4, x_5, x_6)=p(x_1)p(x_2)p(x_3)p(x_4|x_1,x_2,x_3)p(x_5|x_3)p(x_6|x_4,x_5)
\end{equation}
This is more generally expressed in the form:
\begin{equation}
p(\mathbf{x})=\prod_i p(x_i|\textrm{pa}_i)
\label{parent_node_factorization}
\end{equation}
Here $\mathbf{x}$ is a shorthand for all the variables, while $\textrm{pa}_i$ is short for the list of \textit{parent nodes} of $x_i$, i.e. nodes which have edges pointing to $x_i$. If there are no parent nodes, this is simply $p(x_i)$.

\subsection{Example: Polynomial regression}
Let's take a look at a more practical model: Bayesian polynomial regression. Here, we have a set of $N$ observation points:
\begin{equation}
\mathbf{x}=(x_1,\cdots,x_n)^t,\quad \mathbf{t}=(t_1,\cdots,t_n)^t
\end{equation}
We think of the $t$'s as being dependent on the $x$'es: $t=t(x)$. Note that is usually the case in regression models, only the $t$'s are considered random.

\begin{figure}
\centering
\tikz{ %
        \node[latent] (w) {$\mathbf{w}$} ; %
        \node[latent, below= of w] (t3) {$t_3$} ; %
        \node[latent, left= of t3] (t2) {$t_2$} ; %
        \node[latent, left= of t2] (t1) {$t_1$} ; %
		\node[const, right= of t3] (cdots) {$\cdots$} ; %
        \node[latent, right= 2 of t3] (tN) {$t_N$} ; %
        \edge {w} {t1}; \edge {w} {t2}; \edge {w} {t3};
        \edge {w} {tN};   
      }
\caption{Bayesian regression model}
\label{graph:bayesian_regression}
\end{figure}

The model consists of a series of weights, together denoted $\mathbf{w}$. Since the model is Bayesian, this is a random variable. Now, each $t_i$ depends on $x_i$ and $\mathbb{w}$:
\begin{equation}
t_i=t_i(x_i,\mathbf{w})
\end{equation}
But remember that $x_i$ is not random. So, at its heart, the random variables of the model depend on each other as shown in the graph of figure \ref{graph:bayesian_regression}

\end{document}