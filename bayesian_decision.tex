\documentclass[12pt, a4paper]{article}
\usepackage{amsmath}
\usepackage{amsfonts}
\usepackage{amsthm}
\usepackage{mathtools}
\newtheorem{theorem}{Theorem}[section]
\newtheorem{definition}{Definition}[section]
\numberwithin{equation}{section}

\title{Bayesian decision theory}
\author{Kristian Wichmann}

\begin{document}

\maketitle

\section{Loss function}
In order to make a decision based on a posterior distribution, a weighing of different is needed. One such way is a \textit{loss function}, i.e. a function that quantifies how bad a given outcome is. Three simple loss functions are outlined below. Each one leads to one of the traditional measures of centrality for a distribution.

\subsection{All-or-nothing loss function}
This type of loss function treats all wrong outcomes as equally bad. Only the correct outcome has value. It is therefore sometimes called an '1-0' loss function. If the actual outcome is $x_0$

\end{document}