\documentclass[12pt, a4paper]{article}
\usepackage{amsmath}
\usepackage{amsfonts}
\usepackage{amsthm}
\usepackage{mathtools}
\newtheorem{theorem}{Theorem}[section]
\newtheorem{definition}{Definition}[section]
\numberwithin{equation}{section}
\usepackage{pgfplots}
\pgfplotsset{width=10cm,compat=1.9}
\graphicspath{ {img/} }
\DeclareGraphicsExtensions{.png, .jpg}

\title{Exponential families}
\author{Kristian Wichmann}

\begin{document}
\maketitle

These statistical models are of great importance in theoretical statistics.

\section{Definition}
An \textit{exponential family} in $k$ dimensions on a measurable space $(\mathcal{X},\mathbb{E})$ is a parametrized statistical model $\{\nu_\theta|\theta\in\Theta\}$ with the following ingredients:
\begin{itemize}
\item A parameter space $\Theta\subseteq\mathbb{R}^k$ which is an open, convex set.
\item A measurable function $t:\mathcal{X}\rightarrow\mathbb{R}^k$ known as the \textit{canonical sample function}.
\item A $\sigma$-finite measure $\mu$ on $(\mathcal{X},\mathbb{E})$ known as the \textit{base measure} for the family.
\end{itemize}
The probability measures in the family are then given by:
\begin{equation}
\forall A\in\mathbb{E}, \theta\in\Theta:\ \nu_\theta(A)=\frac{1}{c(\theta)}\int_A\exp\left[\theta^t t(x)\right]\ d\mu(x)
\end{equation}
Here, $c(\theta)$ is a normalization constant given by:
\begin{equation}
c(\theta)=\int_A\exp\left[\theta^t t(x)\right]\ d\mu(x)
\end{equation}
We will assume $c(\theta)<\infty$ for all $\theta\in\Theta$. The study of an exponential family often comes down to studying this function, as well as the image measure $t(\mu)$, known as the \textit{structural measure} of the family.

\end{document}