\documentclass[12pt, a4paper]{article}
\usepackage{amsmath}
\usepackage{amsfonts}
\usepackage{amsthm}
\usepackage{mathtools}
\newtheorem{theorem}{Theorem}[section]
\newtheorem{definition}{Definition}[section]
\numberwithin{equation}{section}
\usepackage{pgfplots}
\pgfplotsset{width=10cm,compat=1.9}
\graphicspath{ {img/} }
\DeclareGraphicsExtensions{.png}

\title{t-distributed stochastic neighbor embedding}
\author{Kristian Wichmann}

\begin{document}
\maketitle

\section{Overview}
\textit{t-distributed stochastic neighbor embedding} - or t-SNE for short - is a dimensionality reduction algorithm. It takes a dataset in many dimensions and transforms it into a set with a lower dimensionality, usually two or three.

\subsection{Terminology}
Let $x_1, x_2,\cdots, x_N$ be a dataset in a high dimensional (euclidean) space. And let $\sigma_1, \sigma_2,\cdots,\sigma_N$ be a set of standard deviations of the same size.

\section{First transformation}
Let one of the data points $x_i$ be given. Now consider the conditional probability of drawing $x_j$ from a Gaussian distribution centered at $x_i$ with standard deviation $\sigma_i$:

\end{document}