\documentclass[12pt, a4paper]{article}
\usepackage{amsmath}
\usepackage{amsfonts}
\usepackage{amsthm}
\usepackage{mathtools}
\newtheorem{theorem}{Theorem}[section]
\newtheorem{definition}{Definition}[section]
\numberwithin{equation}{section}
\usepackage{pgfplots}
\pgfplotsset{width=10cm,compat=1.9}
\graphicspath{ {img/} }
\DeclareGraphicsExtensions{.png}

\title{Normal distributions on vector spaces}
\author{Kristian Wichmann}

\begin{document}
\maketitle

\section{Univariate normal distributions}
The \textit{standard normal distribution} is given by the density function:
\begin{equation}
\label{uni_snd}
\phi(z)=\frac{1}{\sqrt{2\pi}}e^{-z^2/2}
\end{equation}
This, like all density functions in this section, is understood to be with respect to the Lebesgue integral on $\mathbb{R}$. A random variable $Z$ that follows this distribution is said to be standard normally distributed, and we write $Z\sim N(0,1)$. It can be shown that $E[Z]=0$ and $\textrm{Var}(Z)=1$.

A general, univariate normal distribution with parameters $\mu$ and $\sigma$, is given by the distribution of $X=\mu+\sigma Z$. If $\sigma\neq 0$, we can invert to find $z=\frac{x-\mu}{\sigma}$. So $\frac{dz}{dx}=\frac{1}{\sigma}$. According to the usual transformation rules, $x$ has the density function:
\begin{equation}
f(x)=\phi\left(\frac{x-\mu}{\sigma}\right)\frac{1}{\sigma}=\frac{1}{\sqrt{2\pi\sigma^2}}e^{-(x-\mu)^2/2\sigma^2}
\end{equation}
We write $X\sim N(\mu,\sigma^2)$. From the usual rules of expectation values and variances, it follows that $E[X]=\mu$ and $\textrm{Var}(X)=\sigma^2$.

\subsection{What if $\sigma=0$?}
The derivation above assumes $\sigma$ to be non-zero. But even if this is the case, $X=\mu$ still has a distribution - it is simply $\mu$ all the time. However, this random variable does not have a density function with respect to the Lebesgue measure. In light of the Radon-Nikodym theorem, this is because the Lebesgue measure does not dominate the probability measure of $X$: $P_X(\{\mu\})=1, m_1(\{\mu\})=0$.

In the multivariate case, we will often run into similar problems.

\section{Random vectors}
Before tackling the multivariate case, we need some basic tools. In this section, we consider vectors of random variables. So a random vector of dimension $n$ is:
\begin{equation}
X=
\begin{pmatrix}
X_1 \\ X_2 \\ \vdots \\ X_n
\end{pmatrix}
\end{equation}
Here, each $X_i$ is a random variable.

\subsection{Variance}
The variance of a $n$-dimensional vector is the $n\times n$ matrix:
\begin{equation}
\textrm{Var}(X)=E[(X-\mu_X)(X-\mu_X)^t]
\end{equation}
Here $\mu_X=E[X]$, i.e. the vector of expectation values of the $X_i$'s. From the usual definitions of variances and covariances between random variables, we see that the diagonal of $\textrm{Var}(X)$ contains the variances of each $X_i$, while the off diagonal elements are the covariances between variables:
\begin{equation}
[\textrm{Var}(X)]_{ij}=\textrm{Cov}(X_i,X_j)
\end{equation}
Due to the symmetry of covariance, this means that $\textrm{Var}(X)$ is a symmetric matrix.

\subsubsection{Variance calculation rules}
Similarly to ordinary random variables, we might calculate the variance matrix as follows:
\begin{align}
\textrm{Var}(X)=&E[(X-\mu_X)(X-\mu_X)^t]=\\
&E(XX^t)-E(X)\mu_X^t-\mu_X E(X)^t+\mu_X\mu_X^t=\\
&E(XX^t)-\mu_X\mu_X^t
\end{align}
Here, we've used the linearity of the expectation value and the definition of $\mu_X$.

Adding a constant vector $b$ does not change the variance, since $E[X+b]=\mu_X+b$:
\begin{equation}
\textrm{Var}(X+b)=E[(X+b-(\mu_X+b))(X+b-(\mu_X+b))^t]=E[(X-\mu_X)(X-\mu_X)^t]
\end{equation}
This is just the variance of $X$.

If $A$ is a constant $m\times n$ matrix and $X$ is an $n$-dimensional random vector, then:
\begin{align}
\textrm{Var}(AX)=&E[(AX-A\mu_X)(AX-A\mu_X)^t]=\\
&E[(A(X-\mu_X))(A(X-\mu_X))^t]=\\
&E[A(X-\mu_X)(X-\mu_X)^t A^t]=\\
&A[(X-\mu_X)(X-\mu_X)^t]A^t
\end{align}
So we have $\textrm{Var}(AX)=A\ \textrm{Var}(X)A^t$.

\subsection{Covariance}
The covariance matrix between two variable vectors $X$ and $Y$ is:
\begin{equation}
\textrm{Cov}(X,Y)=E[(X-\mu_X)(Y-\mu_Y)^t]
\end{equation}
If $X$ has dimension $m$ and $Y$ dimension $n$, then $\textrm{Cov}(X,Y)$ has dimension $m\times n$. Here, the matrix elements reduce to ordinary covariances between $X_i$'s and $Y_j$s:
\begin{equation}
[\textrm{Cov}(X,Y)]_{ij}=\textrm{Cov}(X_i,Y_j)
\end{equation}
This also means, that $\textrm{Cov}(X,Y)=\left(\textrm{Cov}(Y,X)\right)^t$

We note, that the variance could have been defined as a special case of covariance, since $\textrm{Var}(X)=\textrm{Cov}(X,X)$.

\subsubsection{Covariance calculation rules}
Similarly to the rule for variances, we have:
\begin{equation}
\textrm{Cov}(X,Y)=E[XY^t]-\mu_X\mu_Y^t
\end{equation}
The proof is essentially the same.

If $A$ and $B$ are constant matrices of appropriate dimesion, we also have:
\begin{equation}
\textrm{Cov}(AX,BY)=A\ \textrm{Cov}(X,Y)B^t
\end{equation}
Again, the proof is entirely analogous to the corresponding variance formula.

The covariance is bilinear:
\begin{align}
\textrm{Cov}(X+Y,Z)=&\textrm{Cov}(X,Z)+\textrm{Cov}(Y,Z)\\
\textrm{Cov}(X,Y+Z)=&\textrm{Cov}(X,Y)+\textrm{Cov}(X,Z)
\end{align}
This follows from the bilinearity of the ordinary covariance.

\subsubsection{Addiational variance formulas}
Since we noted that $\textrm{Var}(X)=\textrm{Cov}(X,X)$, we may use these rules to derive further properties of variances.

For instance, the variance of a sum:
\begin{align}
\textrm{Var}(X+Y)=&\textrm{Cov}(X+Y,X+Y)=\\
&\textrm{Cov}(X,X)+\textrm{Cov}(X,Y)+\textrm{Cov}(Y,X)+\textrm{Cov}(Y,Y)=\\
&\textrm{Var}(X)+\textrm{Var}(Y)+\textrm{Cov}(X,Y)+\textrm{Cov}(Y,X)
\end{align}
This mirrors the formula for the covariance of sums ordinary random variables, but is complicated by the fact that the vector covariance is not symmetric.

\subsection{Quadratic forms}
If $X$ is an $n$-dimensional random variable and $A$ an $n\times n$ matrix, then the corresponding quadratic form is $Q=X^tAX$. I.e. a scalar. What is the expectation value of the quadratic form? We can use a trick here. Since $Q$ is a scalar, we can trivially write this as a trace:
\begin{equation}
Q=X^tAX=\textrm{tr}(X^tAX)=\textrm{tr}(AXX^t)
\end{equation}
Here, we've used the cyclic property of traces. Now, the expectation value is:
\begin{equation}
E[Q]=E[\textrm{tr}(AXX^t)]=\textrm{tr}(E[AXX^t])=\textrm{tr}(A\ E[XX^t])
\end{equation}
But we know, that $\textrm{Var}(X)=E(XX^t)-\mu_X\mu_X^t$, so $E[XX^t]=\textrm{Var}(X)+\mu_X\mu_X^t$:
\begin{equation}
E[Q]=\textrm{tr}(A(\textrm{Var}(X)+\mu_X\mu_X^t))=\textrm{tr}(A\ \textrm{Var}(X))+\textrm{tr}(A\mu_X\mu_X^t)
\end{equation}
The last term may be rewritten:
\begin{equation}
\textrm{tr}(A\mu_X\mu_X^t)=\textrm{tr}(\mu_X^t A\mu_X)=\mu_X^t A\mu_X
\end{equation}
In the last step we've used that the contents of the parenthesis is a scalar. So all in all:
\begin{equation}
E[X^tAX]=\textrm{tr}(A\ \textrm{Var}(X))+\mu_X^t A\mu_X
\end{equation}

\section{Multivariate standard normal distribution}
The standard normal distribution in $n$ dimensions is the distribution of $n$ independent standard normals. Hence, the density function in $\mathbb{R}^n$ is simply a product of terms like equation \ref{uni_snd}:
\begin{equation}
\phi(z)=\prod_{i=1}^n\frac{1}{\sqrt{2\pi}}e^{-z_i^2/2}=(2\pi)^{-n/2}e^{-||z||^2/2}
\end{equation}
Here $z\in\mathbb{R}^n$, and $||\cdot ||$ is the standard Euclidean norm. If a $n$-dimensional dimensional random vector $Z$ follows this distribution we write $Z\sim N(0,I_n)$, where $I_n$ is the identity matrix in $n$ dimensions. The reason for this is, that the variance matrix of $X$ is equal to $I_n$.

\section{Affine transformations of euclidean spaces}
In order to get to the general, multidimensional normal distribution, we need yet another component:

Let $s:\mathbb{R}^n\rightarrow\mathbb{R}^m$ be a linear transformation. This means that the is an $m\times n$ matrix $A$ so $s(x)=Ax$.

An \textit{affine} transformation $t$ is formed by following this linear map by a translation:
\begin{equation}
t:\mathbb{R}^n\rightarrow\mathbb{R}^m, t(x)=Ax+v
\end{equation}
Here, $v\in\mathbb{R}^m$. Since translations are always bijective, we note that $t$ is bijective iff $A$ is invertible.

Each component of an affine transformation is composed from measurable function - is is understood that we mean with respect to the Borel algebras of each space) - so the affine transformation itself is measurable as well.

\subsection{Transformation properties of the Lebesgue measure}
\label{euclidean_lebesgue_properties}
Recall that the Lebesgue measure in $n$ dimensions $m_n$ is invariant under translation: If $t$ is a translation $t:\mathbb{R}^n\rightarrow\mathbb{R}^n, t(x)=x+x_0$, where $x_0\in\mathbb{R}^n$ then:
\begin{equation}
t(m_n)=m_n
\end{equation}

Also, if $s:\mathbb{R}^n\rightarrow\mathbb{R}^n, x\mapsto Ax$ is an isomorphism, then:
\begin{equation}
\label{lebesgue_transform}
s(m_n)=m_n|\det A^{-1}|
\end{equation}
Combining the two, the formula for affine transformation is the same as for linear ones.

\section{General multivariate normal distribution}
Given an $n$-dimension random vector $Z\sim N(0,I_n)$, we now define a general standard normal as the distribution of the random variable $X=\mu+AZ$, where $A$ is a matrix of dimensions $n\times n$. From our work with random vectors we immediately get that:
\begin{equation}
E(X)=\mu,\quad \textrm{Var}(X)=AA^t=\Sigma
\end{equation}
Here, we've introduced the notation $\Sigma$ for the variance matrix of $X$\footnote{Since $\Sigma$ corresponds to $\sigma^2$ in the univariate case, $\Sigma^2$ would in some sense have been a more logical naming choice, but such is tradition.}.

$A$ plays the roles that $\sigma$ played for the univariate case, and hence we might expect that special care needs to be taken when it is "zero", which will turn out to mean "singular" in the multivariate case.

\subsection{Invertible $A$}
Trying to solve for $Z$ we get $AZ=X-\mu$. If $A$ is invertible we have:
\begin{equation}
Z=A^{-1}(X-\mu)
\end{equation}
We may now write
\begin{equation}
||z||^2=z^t z=\left(A^{-1}(X-\mu)\right)^t\left(A^{-1}(X-\mu)\right)=(X-\mu)\left(A^{-1}\right)^t A^{-1}(X-\mu)
\end{equation}
According to equation \ref{lebesgue_transform} the density function for $X$ is:
\begin{equation}
f(x)=(2\pi)^{-n/2}\textrm{det}(A^{-1})\exp\left[-\frac{1}{2}(X-\mu)\left(A^{-1}\right)^t A^{-1}(X-\mu)\right]
\end{equation}
But the inverse of a transpose is the transpose of the inverse:
\begin{equation}
\left(A^t\right)^{-1}A^{-1}=\left(AA^t\right)^{-1}=\Sigma^{-1}
\end{equation}
So the density function is:
\begin{equation}
f(x)=(2\pi)^{-n/2}\textrm{det}(A^{-1})\exp\left[-\frac{1}{2}(X-\mu)\Sigma^{-1}(X-\mu)\right]
\end{equation}

\subsection{Non-invertible $A$}
If $A$ is not invertible, $X$ does not have a density with respect to the Lebesgue measure in $\mathbb{R}^n$. This is because the rank of $A$ is less than $n$, and so $X$ only takes on values in an affine, proper subspace of $\mathbb{R}^n$. The $n$-dimensional Lebesgue measure of such a space is zero.

\subsection{$A$ from variance $\Sigma$}
Often, we want to specify a univariate normal from a variance $\Sigma$ instead of a transformation matrix $A$. As we saw above, we only need $\Sigma$ to write down the density (if it exists), but what if we want $A$?

To do this, we note that $\Sigma$ is symmetric and positive semi-definite because it is a covariance matrix. So there is an orthogonal matrix $O$ such that $\Sigma=ODO^t$, where $D$ is a diagonal matrix with non-negative diagonal entries. Hence, we can construct another diagonal matrix $D^{\frac{1}{2}}$ where the entries are the square roots of the ones in $D$. Now we obviously have:
\begin{equation}
\Sigma=ODO^t=\underbrace{OD^{\frac{1}{2}}}_{A}\underbrace{D^{\frac{1}{2}}O^t}_{A^t}
\end{equation}
By setting $A=OD^{\frac{1}{2}}$ we have achieved a decomposition of $\Sigma$ that will bring about all the results above (though this is not necessarily the only one). A is sometimes written as the "square root of $\Sigma$": $A=\Sigma^{\frac{1}{2}}$.

\subsubsection{Sphering}
One use of the decomposition $A=OD^{\frac{1}{2}}$ is to gain geometrical insight into multivariate Gaussians by what is called \textit{sphering}.



\section{Orthogonal complement}
Let $V$ be a finite-dimensional vector space with an inner product $\langle\cdot,\cdot\rangle$. Let $U$ be a subspace of $V$. Then we define the \textit{orthogonal complement} of $U$ as:
\begin{equation}
U^\perp=\{v\in V|\forall u\in U: \langle u, v\rangle = 0\}
\end{equation}
\begin{theorem}
$U^\perp$ is a subspace of $V$.
\end{theorem}
\begin{proof}
According to the subspace theorem, we need to show three things:
\begin{itemize}
\item $U^\perp$ is not empty: Clearly $0\in U^\perp$.
\item Closed under addition: If $v_1,v_2\in U^\perp$, then for all $u\in U^\perp$:
\begin{equation}
\langle v_1+v_2,u\rangle=\langle v_1,u\rangle + \langle v_2,u\rangle = 0 
\end{equation}
\item Closed under scalar multiplication: If $v\in U^\perp$ and $c\in\mathbb{R}$ then for all $u\in U^\perp$:
\begin{equation}
\langle cv,u\rangle = c\langle v,u\rangle = 0
\end{equation}
\end{itemize}
\end{proof}

Since the only vector perpendicular to itself is $0$, we further conclude that $U\cap U^\perp=\{0\}$.

\begin{theorem}
If $e_1, e_2,\ldots,e_m$ is an orthonormal basis for $U$, then for any $v\in V$:
\begin{equation}
v-\sum_{i=1}^m\langle v,e_i\rangle e_i\in U^\perp
\end{equation}
\end{theorem}
\begin{proof}
Let $u\in U$. Then we can write $u=\sum_{j=1}^m\lambda_j e_j$ for some coefficients $\lambda_j$. Now calculate the inner product with the vector above:
\begin{equation}
\langle v-\sum_{i=1}^m\langle v,e_i\rangle e_i\ ,\ \sum_{j=1}^m\lambda_j e_j\rangle=\sum_{i=j}^m\lambda_j\langle v,e_j\rangle-\sum_{i=1}^m\sum_{j=1}^m\lambda_j\langle v,e_i\rangle\langle e_i,e_j\rangle
\end{equation}
Since $\langle e_i,e_j\rangle=\delta_{ij}$ this vanishes.
\end{proof}

This means that we may write any $v\in V$ as a sum of vectors from $U$ and $U^\perp$ respectively:
\begin{equation}
\label{u_uperp}
v=\underbrace{\sum_{i=1}^m\langle v,e_i\rangle e_i}_{\in U}\ +\ \underbrace{v-\sum_{i=1}^m\langle v,e_i\rangle e_i}_{\in U^\perp}
\end{equation}

\begin{theorem}
The decomposition into elements from $U$ and $U^\perp$ from equation \ref{u_uperp} is unique.
\end{theorem}
\begin{proof}
Let $v=u_1+u^\perp_1$ and $v=u_2+u^\perp_2$ be two such decompositions. Then $u_1+u^\perp_1=u_2+u^\perp_2$ and hence $u_1-u_2=u^\perp_2-u^\perp_1$. But this means that this vector is a member of both $U$ and $U^\perp$, and hence it must be $0$. This means $u_1=u_2$ and $u^\perp_1=u^\perp_2$.
\end{proof}

\subsection{The orthogonal projection}
The previous section motivates the following:
\begin{definition}
Let $V$ be a finite-dimensional inner product vector space and $U$ a subspace of $V$. The orthogonal projection from $V$ onto $U$ is the map $p:V\rightarrow V$ which satisfies:
\begin{equation}
\forall v\in V:\quad p(v)\in U,\quad v-p(v)\in U^\perp
\end{equation}
\end{definition}
As we see, one could also define the co-domain of $p$ to be $U$. Usually, the distinction will not matter much.

\begin{theorem}
The orthogonal projection operator is linear.
\end{theorem}
\begin{proof}
We need to show additivity and homogeneity:
\begin{itemize}
\item Additivity: Let $v_1,v_2\in V$. Then $p(v_1)+p(v_2)\in U$ and:
\begin{equation}
v_1-p(v_1)+v_2-p(v_2)=v_1+v_2-(p(v_1)+p(v_2))\in U^\perp
\end{equation}
Adding the two we get $v_1+v_2$. So $p(v_1+v_2)=p(v_1)+p(v_2)$.
\item Homogeneity. Let $v\in V$ and $c\in\mathbb{R}$. Then $cp(v)\in U$ and $c(v-p(v))=cv-cp(v)\in U^\perp$. Adding the two we get $cv$, so $p(cv)=cp(v)$. 
\end{itemize}
\end{proof}

\begin{theorem}
The orthogonal projection operator $p:V\rightarrow V$ is idempotent. I.e. $p\circ p=p$.
\end{theorem}
\begin{proof}
Let $v\in V$. Then $p(v)\in U$. But this means that the decomposition of $p(v)$ is $p(v)+0$. So $p\circ p(v)=p(v)$.
\end{proof}

\section{Lebesgue measures on vector spaces}

\subsection{Coordinate maps}
Let $V$ be a finite-dimensional vector space of dimension $n$. Our question is, if we can turn $V$ into a measure space in a natural way. Since we know that $V$ is isomorphic to $\mathbb{R}^n$, it makes sense to tweak the usual Lebesgue measure in $N$ dimensions:

Let $e_1,e_2,\ldots,e_n$ be a basis for $V$. Then we can define the \textit{coordinate map} as follows:
\begin{equation}
\phi:\mathbb{R}^n\rightarrow V,\quad
\begin{pmatrix}
x_1	\\	x_2	\\ \vdots	\\ x_n
\end{pmatrix}
\mapsto\sum_{i=1}^n x_i e_i
\end{equation}
This is obviously an isomorphism. Specifically, it is invertible with inverse $\phi^{-1}: V\rightarrow\mathbb{R}^n$.

The coordinate map depends on the chosen basis. If we had chosen another basis $e^*_1,e^*_2,\ldots,e^*_n$ we would get another isomorphism $\phi^*$.

\subsection{Borel algebra on $V$}
We can now use $\phi^{-1}$ to induce a $\sigma$-algebra on $V$. Set $\mathbb{B}_V$ to the smallest $\sigma$-algebra that makes $\phi^{-1}$ measurable when $\mathbb{R}^n$ is equipped with the Borel algebra $\mathbb{B}_n$. We call $\mathbb{B}_V$ the \textit{Borel algebra on $V$}.

At first this object seems to depend of the choice of basis for $V$. But it turns out that the use of definite article in the definition is justified:
\begin{theorem}
If $e_1,e_2,\ldots,e_n$ and $e^*_1,e^*_2,\ldots,e^*_n$ are bases for $V$, then the induced $\sigma$-algebra $\mathbb{B}_V$ and $\mathbb{B}^*_V$ is the same thing.
\end{theorem}
\begin{proof}
We know that $\phi^{-1}$ is $\mathbb{B}_V-\mathbb{B}_n$ measurable by definition. We have:
\begin{equation}
(\phi^*)^{-1}=(\phi^*)^{-1}\circ\textrm{id}_V=(\phi^*)^{-1}\circ(\phi\circ\phi^{-1})=\left((\phi^*)^{-1}\circ\phi\right)\circ\phi^{-1}
\end{equation}
$\left((\phi^*)^{-1}\circ\phi\right)$ is a linear operator on $\mathbb{R}^n$ and so according to section \ref{euclidean_lebesgue_properties} is measurable. So $(\phi^*)^{-1}$ must be $\mathbb{B}_V-\mathbb{B}_n$-measurable. Since $\mathbb{B}_V^*$ is the smallest $\sigma$-algebra to make $(\phi^*)^{-1}$ $\mathbb{B}_V-\mathbb{B}_n$-measurable, we must have $\mathbb{B}^*_V\subseteq\mathbb{B}_V$.

But by a totally symmetric argument, we must also have $\mathbb{B}_V\subseteq\mathbb{B}^*_V$. Hence $\mathbb{B}_V=\mathbb{B}^*_V$.
\end{proof}

It turns out, that $\phi$ must be measurable too. This is a direct consequence of the pipeline lemma.

\begin{theorem}
Given two finite-dimensional vector spaces $V$ and $W$, then:
\begin{equation}
\mathbb{B}_{V\times W}=\mathbb{B}_V\otimes\mathbb{B}_W
\end{equation}
\end{theorem}
\begin{proof}
Let $e_1, e_2,\ldots, e_n$ be a basis for $V$ with corresponding coordinate map $\phi$. And $f_1, f_2,\ldots, f_m$ a basis for $W$ with corresponding coordinate map $\psi$. Then $(e_1, 0), (e_2, 0),\ldots,(e_n,0),(0, f_1), (0, f_2),\ldots,(0,f_m)$ is a basis for $V\times W$. The corresponding coordinate map is:
\begin{equation}
\phi\times\psi: (x_1, x_2,\ldots,x_{n+m})\mapsto\left(\sum_{i=1}^n x_i e_i,\sum_{j=1}^m x_{n+j}f_j\right)
\end{equation}
The inverse is $(\phi\times\psi)^{-1}=\phi^{-1}\times\psi^{-1}$. Since $\phi^{-1}$ is $\mathbb{B}_V-\mathbb{B}_n$-measurable and $\psi^{-1}$ is $\mathbb{B}_W-\mathbb{B}_m$-measurable, $\phi^{-1}\times\psi^{-1}$ must be $\mathbb{B}_V\otimes\mathbb{B}_W-\mathbb{B}_n\otimes\mathbb{B}_n$-measurable. But $\mathbb{B}_n\otimes\mathbb{B}_m=\mathbb{B}_{n+m}$. Since $\mathbb{B}_{V\times W}$ is the smallest $\sigma$-algebra to make $\phi^{-1}\times\psi^{-1}$ measurable, we must have $\mathbb{B}_{V\times W}\subseteq\mathbb{B}_V\otimes\mathbb{B}_W$.

On the other hand, consider the projection operators:
\begin{align}
\pi_V:& V\times W\rightarrow V, (v,w)\mapsto v\\
\pi_n:& \mathbb{R}^{n+m}\rightarrow\mathbb{R}^n, (x_1,\ldots,x_{n+m})\mapsto(x_1,\ldots,x_n)
\end{align}
Now consider $\pi_V\circ(\phi\times\psi)$. Applied to an $x\in\mathbb{R}^{n+m}$ we have:
\begin{equation}
\pi_V\circ(\phi\times\psi)(x)=\pi_v\left(\left(\sum_{i=1}^n x_i e_i,\sum_{j=1}^m x_{n+j}f_j\right)\right)=\sum_{i=1}^n x_i e_i
\end{equation}
But this is the same as:
\begin{equation}
\phi\circ\pi_1(x)=\phi\left((x_1,\ldots x_n)\right)=\sum_{i=1}^n x_i e_i
\end{equation}
So $\pi_V\circ(\phi\times\psi)=\phi\circ\pi_1$. Now apply $\phi^{-1}\times\psi^{-1}$ from the right to get:
\begin{equation}
\pi_V=\phi\circ\pi_1\circ(\phi^{-1}\times\psi^{-1})
\end{equation}
Since all the three functions on the right side are measurable, $\pi_V$ must be $\mathbb{B}_{V\times W}-\mathbb{B}_V$-measurable. By a similar argument the corresponding projection operator $\pi_W: V\times W\rightarrow W$ is $\mathbb{B}_{V\times W}-\mathbb{B}_W$-measurable. Since $\mathbb{B}_V\otimes\mathbb{B}_W$ is the smallest $\sigma$-algebra to make both $\pi_V$ and $\pi_W$ measurable, we must have: $\mathbb{B}_V\otimes\mathbb{B}_W\subseteq\mathbb{B}_{V\times W}$.
\end{proof}

\subsection{Lebesgue measures on $V$}
We now want to define a measure on the measurable space $(V,\mathbb{B}_V)$. If $e_1, e_2,\ldots, e_n$ is a basis for $V$, we will use the associated coordinate map $\phi$ to define a measure:
\begin{equation}
\lambda_V=\phi(m_n)
\end{equation}
Here, $m_n$ is the usual Lebesgue measure in $n$ dimensions. The problem is, that this measure depends on the chosen basis! Consider another basis $e^*_1, e^*_2,\ldots,e^*_n$ and associated coordinate map $\phi^*$. Then the measure is:
\begin{equation}
\lambda^*_V=\phi^*(m_n)=(\phi\circ\phi^{-1})\circ\phi^*(m_n)=\phi\circ(\phi^{-1}\circ\phi^*(m_n))
\end{equation}
Now $\phi^{-1}\circ\phi^*$ is an isomorphism $\mathbb{R}^n\rightarrow\mathbb{R}^n$, so according to section \ref{euclidean_lebesgue_properties}, there is a constant $c$ such that $(\phi^{-1}\circ\phi^*(m_n))=c m_n$. So:
\begin{equation}
\lambda^*_V=c\phi(m_n)=c\lambda_V
\end{equation}

So while there are many Lebesgue measures on $V$ they only differ from each other by a constant factor. This means that they all agree on what constitutes a null set, and on which functions are integrable. They disagree on the integral, but agree on whether it is finite or not. The also agree on whether a measure $\mu$ has a density with respect to $\lambda_V$ or not.


\end{document}