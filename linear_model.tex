\documentclass[12pt, a4paper]{article}
\usepackage{amsmath}
\usepackage{amsfonts}
\usepackage{amsthm}
\usepackage{mathtools}
\usepackage{MnSymbol}
\newtheorem{theorem}{Theorem}

\title{The Linear Model}
\author{Kristian Wichmann}

\begin{document}

\maketitle

The \textit{linear model} is a theoretical framework that unifies a number of statistical concepts, like ANOVA and regression.

\section{Derivatives and linear algebra}
We will need a few results concerning derivatives of linear algebra expressions. Consider a linear function:
\begin{equation}
f: \mathbb{R}^n\rightarrow\mathbb{R}, f(\beta)=A\beta=
\begin{pmatrix}
a_1 & a_2 & \cdots & a_n
\end{pmatrix}
\begin{pmatrix}
\beta_1 \\ \beta_2 \\ \vdots \\ \beta_n
\end{pmatrix}
\end{equation}
Here, $A\in\mathbb{R}^{1\times n}$ and $\beta\in\mathbb{R}^{n\times 1}$, so in other words:
\begin{equation}
f(\beta)=a_1\beta_1+a_2\beta_2+\cdots a_n\beta_n
\end{equation}
The (multidimensional) derivative is therefore:
\begin{equation}
\label{scalarproductdif}
\frac{\partial f}{\partial \beta}=\nabla_\beta f=
\begin{pmatrix}
a_1 \\ a_2 \\ \vdots \\ a_n
\end{pmatrix}
=A^t
\end{equation}
Similarly, consider a quadratic form in $\beta$:
\begin{equation}
g: \mathbb{R}^n\rightarrow\mathbb{R}, g(\beta)=\beta^t A\beta =
\begin{pmatrix}
\beta_1 & \beta_2 & \cdots & \beta_n
\end{pmatrix}
\begin{pmatrix}
a_{11} & a_{12} & \cdots & a_{1n} \\
a_{21} & a_{22} & \cdots & a_{2n} \\
\vdots & \vdots & \ddots & \vdots \\
a_{n1} & a_{n2} & \cdots & a_{nn}
\end{pmatrix}
\begin{pmatrix}
\beta_1 \\ \beta_2 \\ \vdots \\ \beta_n
\end{pmatrix}
\end{equation}
Here, $A\in\mathbb{R}^{n\times n}$ and $\beta\in\mathbb{R}^{n\times 1}$. Furthermore, $A$ is assumed to be symmetric, such that $a_{ij}=a_{ji}$. Multiplying out, this means that:
\begin{equation}
g(\beta)=\sum_{i=1}^n\sum_{j=1}^n\beta_i a_{ij}\beta_j
\end{equation}
Differentiating with respect to $\beta_k$ only terms where $i=k$ or $i=j$ will contribute. However, the case $i=j=k$ is distinct. So, when $i=k$ we get the contribution $a_{kj}\beta_j$. When $j=k$ we get $\beta_i a_{ij}$. And when $i=j=k$ we get $2a_{kk}\beta_k$. All in all, when summing up, we get two of each $a$-$\beta$ set (because of the symmetry of $A$). So:
\begin{equation}
\frac{\partial g}{\partial\beta_k}=2\sum_{i=1}^n a_{ik}\beta_i
\end{equation}
Or more compactly:
\begin{equation}
\label{quadformdif}
\frac{\partial g}{\partial \beta}=\nabla_\beta g=2A\beta
\end{equation}

\section{Ordinary least squares estimation (OLS)}

\subsection{Statement of the problem}
The general problem is this: We wish to model a linear relationship between a response variables $Y$ and $p$ predictor variables $X_1, X_2,\cdots X_p$. In other words:
\begin{equation}
\label{linear1}
Y=\beta_1 X_1 + \beta_2 X_2 + \cdots + \beta_p X_p+\epsilon
\end{equation}
Here, the $\beta$'s are the coefficients corresponding to the $X$'s. The random term $\epsilon$ is known as the \textit{error term} and represents the deviations from the exact model. Since it is a random variable, so is $Y$. Now, assume that we have $n$ realizations (data points), so that $y_i$ corresponds to $x_{i1}, x_{i2},\cdots x_{ip}$. In matrix form equation $(\ref{linear1})$ now becomes: 
\begin{equation}
\label{linear2}
y=X\beta+\epsilon
\end{equation}
Here, $y\in\mathbb{R}^{n\times 1}, X\in\mathbb{R}^{n\times p}$ and $\beta\in\mathbb{R}^{p\times 1}$:
\begin{equation}
y = 
\begin{pmatrix}
	y_1 \\ y_2 \\ \vdots \\ y_n
\end{pmatrix},\quad
X =
\begin{pmatrix}
	x_{11} 	& x_{12} 	& \ldots	& x_{1p} \\
	x_{21} 	& x_{22} 	& \ldots	& x_{2p} \\
	\vdots	& \vdots	& \ddots	& \vdots \\
	x_{n1} 	& x_{n2} 	& \ldots	& x_{np} \\
\end{pmatrix},\quad
\beta =
\begin{pmatrix}
	\beta_1	\\ \beta_2	\\ \vdots\\	\beta_p 
\end{pmatrix},\quad
\epsilon = 
\begin{pmatrix}
	\epsilon_1 \\ \epsilon_2 \\ \vdots \\ \epsilon_n
\end{pmatrix}
\end{equation}
$X$ is known as the \textit{design matrix}. Given $y$ and $X$, we seek the best fit for $\beta$.

\subsection{Least squares}
There's a number of criteria one could use to pick the best fitting $\beta$. Here, we will search for the one that minimizes the square of the differences in predicted and actual $y$ values. When predicting, we can't include the error term, and so the predicted values for a set of parameters $\hat{\beta}$ are simply:
\begin{equation}
y=X\hat{\beta}
\end{equation}
The squared difference is:
\begin{align*}
||y-X\hat{\beta}||^2 &=(y-X\hat{\beta})^t(y-X\hat{\beta})=(y^t-\hat{\beta}^t X^t)(y-X\hat{\beta})\\
&=y^t y - 2y^t X\hat{\beta} + \hat{\beta}^t X^t X\hat{\beta}
\end{align*}
Taking the derivative with respect to $\beta$ we can now use equations $(\ref{scalarproductdif})$ and $(\ref{quadformdif})$ to yield:
\begin{equation}
2X^t y - 2X^t X\hat{\beta}
\end{equation}
Since we're looking for a minimum, this vector should be equal to zero:
\begin{equation}
\label{normaleq}
2X^t y - 2X^t X\hat{\beta}=0\Leftrightarrow\hat{\beta}=(X^t X)^{-1}X^t y=\beta+(X^t X)^{-1}X^t\epsilon
\end{equation}
Here it has been assumed that $X^t X$ is invertible. If $X^t X$ is not invertible, we have a case of \textit{perfect (multi)collinearity}. The equations in \ref{normaleq} are known as the \textit{normal equations} for the model. Inserting into equation $(\ref{linear2})$ we get the corresponding predicted $y$-values, also denoted by a hat:
\begin{equation}
\hat{y}=X\hat{\beta}=\underbrace{X(X^t X)^{-1}X^t}_{H}y
\end{equation}
The matrix $H=X(X^t X)^{-1}X^t$ is often called the \textit{hat matrix}, since it puts the hat on the $y$'s. The hat matrix can also be used to find \textit{residuals}, i.e. the difference between actual and predicted $y$-values:
\begin{equation}
e=y-\hat{y}=y-Hy=\underbrace{(I-H)}_{M} y
\end{equation}

\subsection{Properties of the OLS estimator}
First of all we note, that the estimation is a linear function of the $y$ values.

The estimated value of $\beta$ according to OLS is:
\begin{equation}
\hat{\beta}=(X^t X)^{-1}X^t y=(X^t X)^{-1}X^t(X\beta+\epsilon)
\end{equation}
Here, we have re-inserted equation \ref{linear2}. We may now consider the expected value:
\begin{equation}
\label{ols_var}
E[\hat{\beta}]=\beta+(X^t X)X^t E[\epsilon]
\end{equation}



\section{Geometric picture}
It is useful to adapt the picture of the columns of $X$ spanning a $p$-dimensional hyperplane in $n$-dimensional space. $y$ is then a vector, and $X\hat{\beta}$ is found by projecting $y$ onto the hyperplane; The corresponding point is exactly the one that minimizes the distance between $y$ (as a point) and the hyperplane.

\subsection{Projection operators}
A linear map that is symmetric and idempotent is called a \textit{projection}. A matrix corresponding to such a mapping is a projection matrix.

\begin{theorem}
The hat matrix $H$ is a projection matrix.
\end{theorem}
\begin{proof}
We need to show that $H$ is symmetric and idempotent. Symmetry:
\begin{equation}
X(X^t X)^{-1}X^t)^t=X\left[(X^t X)^{-1}\right]^t X^t
\end{equation}
But the transpose of an inverse is the same as the inverse of the transpose, so:
\begin{equation}
\left[(X^t X)^{-1}\right]^t=\left[(X^t X)^t\right]^{-1}=(X^t X)^{-1}
\end{equation}
This proves the symmetry of $H$. Idempotency:
\begin{equation}
H^2=\left[X(X^t X)^{-1}X^t\right]^2=X(X^t X)^{-1}X^tX(X^t X)^{-1}X^t=X(X^t X)^{-1}X^t=H
\end{equation}
\end{proof}

This also turns out to be true for the matrix used to find residuals:

\begin{theorem}
The matrix $M=I-H$ is a projection matrix.
\end{theorem}
\begin{proof}
Symmetry follows from the symmetry of $H$. Idempotency:
\begin{equation}
M^2=(I-H)^2=I^2+H^2-2H=I+H-2H=I-H=M
\end{equation}
\end{proof}

\section{The Gauss-Markov theorem}
So far, we have considered only the ordinary least squares (OLS) estimator of the vector $\beta$. But clearly it is not the only possibility. What is the justification for picking this particular estimator? The answer lies in the \textit{Gauss-Markov theorem}. According to this theorem, under certain conditions, the OLS is the best linear unbiased estimator. This is often abbreviated to \textit{BLUE}. Let's examine the meaning of this.

\subsection{Linear and unbiased}
We already know that the OLS estimator is linear in terms of the $y$'s.

Recall, that an estimator is \textit{unbiased} if its expectation value is the true value. Here it means:
\begin{equation}
E[\hat{\beta}]=\beta
\end{equation}
From equation \ref{ols_var} we know, that this is true exactly when the expectation value of $\epsilon$ is zero.

\subsection{'Best'}
In this context, "best" means having the smallest possible variance. We could express this by requiring every estimator element of the $\hat{\beta}$ vector to have a minimal variance. But we will go further than this: Let $\hat{\gamma}=\sum_{i=1}^p c_i\hat{\beta}_i=C\hat{\beta}$ be an arbitrary linear combination of the predictors. Then the variance of every such expression should be minimal. According to the usual rules of calculating variance:
\begin{equation}
\textrm{var}(\hat{\gamma})=\textrm{var}(C\hat{\beta})=C\textrm{var}(\hat{\beta})C^t
\end{equation}
Consider another estimator $\tilde{\beta}$ which has a covariance matrix of:
\begin{equation}
\textrm{var}(\tilde{\beta})=\textrm{var}(\hat{\beta})+\Delta
\end{equation}
Here $\Delta$ is the deviation from our proposed best estimator. The variance of a linear combination of the tilde estimator is:
\begin{equation}
\textrm{var}(\tilde{\gamma})=C\textrm{var}(\tilde{\beta})C^t=C(\textrm{var}(\hat{\beta})+\Delta)C^t=\textrm{var}(\tilde{\beta})+\underbrace{C\Delta C^t}
\end{equation}
Hence $\hat{\beta}$ is the best estimator if and only if the underbraced quantity is always positive, except when $C=0$. In other words, exactly when $\Delta$ is positive definite.

\subsection{The theorem}

\begin{theorem}
(Gauss-Markov) Given a linear model with design matrix $X$, responses $y$, and true parameters $\beta$. Assume the following three conditions for the error terms $\epsilon_i$ are met:
\begin{itemize}
\item The expected value is zero: $E[\epsilon_i]=0$.
\item The variance of the error terms are finite and constant: $\textrm{var}(\epsilon_i)=\sigma^2<\infty$. This is known as homoscedasticity.
\item The error terms are pairwise uncorrelated: $\textrm{cov}(\epsilon_i,\epsilon_j)=0, i\neq j$.
\end{itemize}
Then, the OLS estimator $\hat{\beta}=(X^t X)^{-1}X^t y$ is BLUE.
\end{theorem}

\begin{proof}
Let $\tilde{\beta}=Cy$ be another unbiased linear estimator of $\beta$. We may  then write the matrix $C$ as $(X^t X)^{-1}X^t+D$, where $D$ is the deviation from the OLS estimator. Then we may calculate the expected value:
\begin{equation}
E[\tilde{\beta}]=E[Cy]=E[(X^t X)^{-1}X^t+D)(X\beta+\epsilon)]=(X^t X)^{-1}X^t+D)(X\beta+E[\epsilon])
\end{equation}
By the first assumption this is:
\begin{equation}
((X^t X)^{-1}X^t+D)X\beta=(X^t X)^{-1}X^t X\beta+DX\beta=\beta+DX\beta
\end{equation}
Since $\tilde{\beta}$ is an unbiased estimator, we must have $DX=0$. Now, let's compute the variance:
\begin{equation}
\textrm{var}(\tilde{\beta})=\textrm{var}(Cy)=C\textrm{var}(y)C^t
\end{equation}
Here, we've used a property of variances. By the homoscedasticity assumptions, this is simply:
\begin{equation}
\sigma^2 CC^t=\sigma^2((X^t X)^{-1}X^t+D)((X^t X)^{-1}X^t+D)^t
\end{equation}
Since $X^t X$ is symmetric, so is the inverse, so $((X^t X)^{-1}X^t+D)^t=X(X^t X)^{-1}+D^t$. So we get:
\begin{equation}
\sigma^2((X^t X)^{-1}X^t+D)(X(X^t X)^{-1}+D^t)
\end{equation}
Ignoring the $\sigma^2$ factor for a while, this is:
\begin{equation}
(X^t X)^{-1}X^tX(X^t X)^{-1}+(X^t X)^{-1}X^tD^t+DX(X^t X)^{-1}+DD^t
\end{equation}
But since we just concluded $DX=0$ the two middle terms vanish (since $X^tD^t=(DX)^t=0$). So, reinstating the $\sigma^2$, the variance is
\begin{equation}
\textrm{var}(\tilde{\beta})=\sigma^2(X^t X)^{-1}+\sigma^2 DD^t
\end{equation}
The first term is what we would get without the $D$ term, and is therefore the variance of the OLS estimator. $DD^t$ is a positive definite matrix, and hence according to the section above, $\hat{\beta}$ is the least variance estimator.
\end{proof}

Note that no assumptions of independence, identical distribution or normality is assumed of the error terms.

\subsection{Omitted variable bias}
Assume we have forgotten, missed or simply not had access to a (set of) important predictor variables $z_1, z_2,\ldots, z_n$. The parameters corresponding to these are called $\gamma$, and we may now rewrite the model:
\begin{equation}
y=X\beta+\underbrace{Z\gamma+\delta}_{\epsilon}
\end{equation}
Here, $\delta$ is the error terms associated with the new variables. We will assume that since the missing variables are important/good, the expectation value of this error is zero: $E[\delta]=0$. The error term of the original model is now the underbraced part of the equation.

\section{Abstract definition of a linear model}
This section will deal with the linear model in its most abstract form.

Let $V$ be a vector space of finite dimension $N$. To specify a linear model we need two ingredients:
\begin{itemize}
\item A subspace $L\subset V$. Do note that we require $L$ to be a proper subset of $V$, i.e. $\textrm{dim}L<N$. This subspace is known as the \textit{mean value subspace}.
\item An inner product $\langle\cdot,\cdot\rangle$ on $V$.
\end{itemize}
The inner product induces a family of inner products $\llangle\cdot,\cdot\rrangle_{\sigma^2}$ parametrized by $\sigma^2>0$:
\begin{equation}
\llangle\cdot,\cdot\rrangle_{\sigma^2}=\frac{\langle\cdot,\cdot\rangle}{\sigma^2}
\end{equation}
These inner products are known as \textit{precisions}. While they do not agree on distances, the precisions do agree on orthogonality.

The linear model


\end{document}