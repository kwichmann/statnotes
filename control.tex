\documentclass[12pt, a4paper]{article}
\usepackage{amsmath}
\usepackage{amsfonts}
\usepackage{amsthm}
\usepackage{mathtools}
\newtheorem{theorem}{Theorem}[section]
\newtheorem{definition}{Definition}[section]
\numberwithin{equation}{section}
\usepackage{pgfplots}
\pgfplotsset{width=10cm,compat=1.9}
\graphicspath{ {img/} }
\DeclareGraphicsExtensions{.png, .jpg}

\title{Control Theory}
\author{Kristian Wichmann}

\begin{document}
\maketitle

\section{Control and error}
\emph{Control theory} deals with strategies for keeping a quantity at a constant level in a dynamic system. In mathematical terms we try to keep a quantity $y(t)$ at a constant level $y_r$ over time $t$.

To achieve this goal, a \emph{controller} will affect the system at all time. This will generally be based on the \emph{error}, i.e. the current deviation from the desired level:
\begin{equation}
e(t)=y_r-y(t)
\end{equation}

\section{P-control}
\emph{P-control} is the case where the controller correction $u$ is proportional (hence the P) to the error:
\begin{equation}
u(t)=ke(t)=k(y_r-y(t))
\end{equation}

\subsection{Example: Anaesthesia}
Surgery is performed on a patient. During the procedure, it is desirable to keep the blood concentration of anaesthetic $y(t)$ at a constant level $y_r$. Without control, the concentration follows the following differential equation:
\begin{equation}
\frac{dy}{dt}=-ay
\end{equation}
I.e. it will decay exponentially from a starting concentration $y_0=y(0)$:
\begin{equation}
y(t)=y_0\cdot e^{-at}
\end{equation}
We now add the control term:
\begin{equation}
\frac{dy}{dt}=-ay+u=-ay+k(y_r-y(t))=ky_r-(a+k)y
\end{equation}
This is a differential equation of the form:
\begin{equation}
\frac{dy}{dy}=-b+ay
\end{equation}
Which has the general solution:
\begin{equation}
y(t)=-\frac{b}{a}+c\cdot e^{-at}
\end{equation}
Here, this means:
\begin{equation}
y(t)=\frac{ky_r}{a+k}+c\cdot e^{(a+k)t}
\end{equation}
With the boundary condition that $y(0)=0$ we can determine $c$:
\begin{equation}
c=-\frac{ky_r}{a+k}
\end{equation}
We can now write the solution as:
\begin{equation}
y(t)=\frac{ky_r}{a+k}-\frac{ky_r}{a+k}e^{(a+k)t}
\end{equation}
So the error is:
\begin{equation}
e(t)=y_r-y(t)=y_r-\frac{ky_r}{a+k}+\frac{ky_r}{a+k}e^{(a+k)t}
\end{equation}
Expand first term to get common denominator:
\begin{align}
e(t)&=\frac{y_r(a+k)}{a+k}-\frac{ky_r}{a+k}+\frac{ky_r}{a+k}e^{(a+k)t}\\
&=\frac{y_r}{a+k}\left[a+k(e^{(a+k)t}-1)\right]
\end{align}
The controller dose is then found by multiplying by $k$:
\begin{equation}
u(t)=\frac{y_r}{a+k}\left[ak+k^2(e^{(a+k)t}-1)\right]
\end{equation}
However, we now see that in the limit $t\rightarrow\infty$ the error is actually not zero, as we would hope for, but instead:
\begin{equation}
\lim_{t\rightarrow\infty}e(t)=y_r\frac{a}{a+k}
\end{equation}

\section{Laplace transforms}
Given a function $f=f(t)$ defined for all positive $t$. Then the Laplace transform of it is defined as:
\begin{equation}
\mathcal{L}[f](s)=\int_0^\infty f(t)e^{-ts}\ dt
\end{equation}
The notation $F(s)$ is often used as a shorthand, and similarly for other functions.

\subsection{Properties of the Laplace transform}
The Laplace transform is linear, since integration is:
\begin{align}
\mathcal{L}[af+bg](s)&=\int_0^\infty\left[af(t)+b(g(t)\right]e^{-ts}\ dt\\
&=a\int_0^\infty f(t)e^{-ts}\ dt+b\int_0^\infty g(t)e^{-ts}\ dt\\
&=a\mathcal{L}[f](s)+b\mathcal{L}[g](s)
\end{align}
Laplace transforming a derivative gives us:
\begin{align}
\mathcal{L}\left[\frac{df}{dt}\right](s)&=\int_0^\infty \frac{df(t)}{dt}e^{-ts}\ dt\\
&=\left[f(t)e^{-ts}\right]_0^\infty-\int_0^\infty f(t)\frac{d}{dt}e^{-ts}\ dt\\
&=-f(0)+s\int_0^\infty f(t)e^{-ts}\ dt\\
&=s\mathcal{L}[f](s)-f(0)
\end{align}
Here partial integration has been used. Note that we have assumed that $f(t)$ grows slower than an exponential for $t\rightarrow\infty$.

Similarly, we can transform an integral:
\begin{align}
\mathcal{L}\left[\int_0^t f(x)\ dx\right](s)&=\int_0^\infty\int_0^t f(x)\ dx\ e^{-ts}\ dt\\
&=\left[\int_0^t f(x)\ dx\cdot\left(-\frac{1}{s}\right)e^{-ts}\right]_0^\infty-\int_0^\infty f(t)\left(-\frac{1}{s}\right)e^{-ts}\ dt\\
&=\frac{1}{s}\mathcal{L}[f](s)
\end{align}
Again, we have made assumptions on the growth speed of the integrand, i.e. this time of the integral of $f$.

\subsection{A few select Laplace transforms}
We consider two specific Laplace transforms in this section. First of a constant:
\begin{align}
\mathcal{L}[k](s)&=\int_0^\infty k\cdot e^{-st}\ dt\\
&=k\int_0^\infty e^{-st}\ dt\\
&=-\frac{k}{s}[e^{-st}]_0^\infty=\frac{k}{s}
\end{align}
And then of the function $te^{at}$:
\begin{align}
\mathcal{L}[te^{at}](s)&=\int_0^\infty te^{at}\cdot e^{-st}\ dt\\
&=\int_0^\infty te^{(a-s)t}\ dt\\
&=\left[t\frac{1}{a-s}e^{(a-s)t}\right]_0^\infty-\int_0^\infty 1\cdot\frac{1}{a-s}e^{(a-s)t}\ dt=\frac{1}{(s-a)^2}
\end{align}

\end{document}
