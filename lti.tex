\documentclass[12pt, a4paper]{article}
\usepackage{amsmath}
\usepackage{amsfonts}
\usepackage{amsthm}
\usepackage{mathtools}
\newtheorem{theorem}{Theorem}[section]
\newtheorem{definition}{Definition}[section]
\numberwithin{equation}{section}
\usepackage{pgfplots}
\pgfplotsset{width=10cm,compat=1.9}
\graphicspath{ {img/} }
\DeclareGraphicsExtensions{.png}

\title{LTI dynamical systems}
\author{Kristian Wichmann}

\begin{document}
\maketitle

\section{Linear, time-independent systems}
Consider a dynamic system with an input $x(t)$ and an output $y(t)$. The time variable $t$ may be \textit{continuous}, i.e. $t\in\mathbb{R}$, or \textit{discrete} i.e. $t\in\mathbb{Z}$. In the following we will start with the continous case and deal with the discrete case in its own section.

The transformation from input to output can be thought of as an operator $\mathcal{F}:x\mapsto y$, sometimes known as a \textit{filter}. We call the dynamical system \textit{linear}, if $\mathcal{F}$ is linear. We call the system \textit{time-independent} if translations in time does not change the dynamics of the system:
\begin{equation}
\mathcal{F}[x](t)=y(t)\Rightarrow\mathcal{F}[x](t-\tau)=y(t-\tau)
\end{equation}
This should hold for all values of $t$ and $\tau$.

We will take special interest in systems that are both linear and time-independent. These are called \textit{LTI} for short.

\subsection{Impulse response function}
Consider an input to an LTI system, which is a delta function impulse at time $t=0$, so that $x(t)=\delta(t)$. The corresponding output $h(t)$ is called the \textit{impulse response function}. Since the system is time-independent, this implies that an input of $x(t)=\delta(t-\tau)$ will have output $h(t-\tau)$.

Consider now a general input $x(t)$. We can trivially rewrite this using a delta function:
\begin{equation}
x(t)=\int_{-\infty}^\infty x(\tau)\delta(t-\tau)\ d\tau
\end{equation}
We can think of this as splitting the input into a continuous series of infinitimal impulses. We can now use the linearity of $\mathcal{F}$ to calculate the output:
\begin{equation}
y(t)=\int_{-\infty}^\infty x(\tau)h(t-\tau)\ d\tau=(x*h)(t)
\end{equation}
We see that we end up with the \textit{convolution} between the input and the impulse responce function. So if we know $h(t)$ we can - in principle at least - always calculate the output.

\subsubsection{Discrete case}
Here the initial input is a Kronecker delta: $x(t)=\delta_{t0}$, with output $h(t)$. So by time-independence an input of $x(t-\tau)$ results in output $h(t-\tau)$. A general input can be rewritten:
\begin{equation}
x(t)=\sum_{\tau\in\mathbb{Z}}x(\tau)\delta_{t\tau}
\end{equation}
So the output is the discrete convolution between $x$ and $h$:
\begin{equation}
y(t)=\sum_{\tau\in\mathbb{Z}}x(\tau)h(t-\tau)
\end{equation}


\end{document}