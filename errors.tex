\documentclass[12pt, a4paper]{article}
\usepackage{amsmath}
\usepackage{amsfonts}
\usepackage{amsthm}
\usepackage{mathtools}
\newtheorem{theorem}{Theorem}[section]
\newtheorem{definition}{Definition}[section]
\numberwithin{equation}{section}
\usepackage{pgfplots}
\pgfplotsset{width=10cm,compat=1.9}
\graphicspath{ {img/} }
\DeclareGraphicsExtensions{.png}

\title{Error functions - classification}
\author{Kristian Wichmann}

\begin{document}
\maketitle

\section{Continous-valued predictors}
Most predictors for classification will be based on a continuous-valued score for each of the two outcomes. The prediction is then made based on these values (usually, the one with the highest score is chosen).

Given the values from such a predictor, we would like to have some quantitative measure of how good the prediction is.

\subsection{Terminology}
Consider classification between $n$ cases. If the true classification for a sample $x$ is $k$, then we define the $n$-dimensional vector $t$ by $t_i=\delta_{ik}$. Let the values made by the predictor be the vector $h$, also a $n$-dimensional vector.

\section{Error functions}

\subsection{Squared error}
In this scheme we use the same error as used for regression - the squared error:
\begin{equation}
\frac{1}{2}||t-h||^2=\frac{1}{2}(t-h)^T(t-h)=\frac{1}{2}\sum_{i=1}^n(t_i-h_i)^2
\end{equation}

\subsection{Cross-entropy error}
This error is based on the log-likelihood of a Bernoulli experiment:
\begin{equation}
\sum_{i=1}^n\left[-t_i\log(h_i)-(1-t_i)\log(1-h_i)\right]
\end{equation}

\section{Forms of the hypothesis $h$}
The hypothesis $h$ is in general the function of some input feature vector $x$, so $h=h(x)$. Here we will consider only hypotheses which depends on the vector $x$ weighted by a vector of weights $w_i$ depending on the outcome $i$. In other words, we only look at $h_i(x)=h_i(w_i^T x)$.

\subsection{Linear hypothesis}
Here, the hypothesis is simply equal to $w_i^T x$, so that:
\begin{equation}
h_i(x)=w_i^Tx
\end{equation}

\subsection{Logistic hypothesis}
Here, the hypothesis is the logistic function of $w_i^T x$. The logistic function is:
\begin{equation}
\sigma(z)=\frac{1}{1+e^{-z}}
\end{equation}
And so, the hypothesis is:
\begin{equation}
h_i(x)=\sigma(w_i^T x)
\end{equation}

\subsection{Softmax hypothesis}
Here, the softmax function is used to normalize the hypothesis, so that they sum to 1. The softmax is defined by:
\begin{equation}
s_i(z)=\frac{e^{z_i}}{\sum_{j=1}^n e^{z_j}}
\end{equation}
And the hypothesis:
\begin{equation}
h_i(x)=s_i(w^T x)
\end{equation}
Here, $w$ is a matrix in which the columns are the weights $w_i$.

\end{document}