\documentclass[12pt, a4paper]{article}
\usepackage{amsmath}
\usepackage{amsfonts}
\usepackage{amsthm}
\usepackage{mathtools}
\newtheorem{theorem}{Theorem}[section]
\newtheorem{definition}{Definition}[section]
\numberwithin{equation}{section}

\title{Association rule data mining}
\author{Kristian Wichmann}

\begin{document}
\maketitle

\section{Transactions and patterns}
Let $\mathcal{I}$ be a set of so-called \textit{transactions}. A \textit{pattern} $t$ is a subset of $\mathcal{I}$. So we might consider a dataset $\mathcal{T}$ with $n$ different transactions and $m$ patterns:
\begin{equation}
\mathcal{I}=\{I_1, I_2,\ldots,I_N\},\qquad \mathcal{T}=\{t_1, t_2,\ldots,t_n\}
\end{equation}
\textit{Association rule data mining} is an unsupervised learning discipline trying to reveal systematics in the database.

\subsection{Support of a pattern}
A pattern $X\subseteq\mathcal{I}$ is said to have a \textit{support} equal to the number of elements of $\mathcal{T}$ in which $X$ is a subset. Support may be specified absolutely or relatively:
\begin{equation}
N_X=|X|,\quad\textrm{supp}X=\frac{X_N}{N}
\end{equation}

\section{Association rules}
An \textit{association rule} takes the form $X\Rightarrow Y$, where $X$ and $Y$ are disjoint patterns.

For instance, consider:
\begin{equation}
\textrm{\{beer\}}\Rightarrow\textrm{\{diapers\}}
\end{equation}
This is an association rule, which says that people who buy beer, typically also purchases diapers. Another (perhaps less surprising) association rule would be:
\begin{equation}
\textrm{\{cheese, ham, bread\}}\Rightarrow\textrm{\{butter\}}
\end{equation}
Note the asymmetry between union and intersection here.

\subsection{Support of an associaton rule}
Like patterns, assocation rules have support associated with them. The support of $X\Rightarrow Y$ is simply equal to the support of $X\cap Y$. Again, this can be specifed absolutely or relatively:
\begin{equation}
N_{X\Rightarrow Y}=N_{X\cup Y}=|X\cap Y|,\quad\textrm{supp}(X\Rightarrow Y)=\frac{N_{X\cup Y}}{N}
\end{equation}

\subsection{Confidence of an associaton rule}
The confidence for the association rule $X\Rightarrow Y$ is defined as:
\begin{equation}
\textrm{conf}(X\Rightarrow Y)=\frac{\textrm{supp}(X\Rightarrow Y)}{\textrm{supp}X}=\frac{\textrm{supp}(X\cup Y)}{\textrm{supp}X}=\frac{N_{X\Rightarrow Y}}{N_X}=\frac{N_{X\cup Y}}{N_Y}
\end{equation}
So the confidence can be seen as a conditional probability: Given $X$, what is the chance of $Y$?

\subsection{Lift of an associaton rule}
The \textit{lift} for the association rule $X\Rightarrow Y$ is defined as:
\begin{equation}
\textrm{lift}(X\Rightarrow Y)=\frac{\textrm{supp}(X\Rightarrow Y)}{\textrm{supp}X\cdot\textrm{supp}Y}
\end{equation}
Usng the definition of relative support lift can also be written:
\begin{equation}
\textrm{lift}(X\Rightarrow Y)=\frac{N_{X\cup Y}/N}{N_X/N\cdot N_Y/N}=\frac{N_{X\cup Y}\cdot N}{N_X\cdot N_Y}
\end{equation}

The interpretation of lift has to do with independence or lack thereof. If lift is equal to 1, this means:
\begin{equation}
1=\frac{\textrm{supp}(X\cup Y)}{\textrm{supp}X\cdot\textrm{supp}Y}\Leftrightarrow\textrm{supp}(X\cup Y)=\textrm{supp}X\cdot\textrm{supp}Y
\end{equation}
Reinterpreting as probabilities, this means:
\begin{equation}
p(X\cap Y)=p(X)\cdot p(Y)
\end{equation}
In other words, it implies independence between $X$ and $Y$. Lower values means negative correlation, and higher values positive correlation.

\subsection{Conviction of an associaton rule}
The \textit{conviction} for the association rule $X\Rightarrow Y$ is defined as:
\begin{equation}
\textrm{conv}(X\Rightarrow Y)=\frac{1-\textrm{supp}Y}{1-\textrm{conf}(X\Rightarrow Y)}
\end{equation}

\end{document}