\documentclass[12pt, a4paper]{article}
\usepackage{amsmath}
\usepackage{amsfonts}
\usepackage{amsthm}
\usepackage{mathtools}
\newtheorem{theorem}{Theorem}[section]
\newtheorem{definition}{Definition}[section]
\numberwithin{equation}{section}

\title{Association rule data mining}
\author{Kristian Wichmann}

\begin{document}
\maketitle

\section{Transactions and patterns}
Let $\mathcal{I}$ be a set of so-called \textit{transactions}. A \textit{pattern} $t$ is a subset of $\mathcal{I}$. So we might consider a dataset $\mathcal{T}$ with $n$ different transactions and $m$ patterns:
\begin{equation}
\mathcal{I}=\{I_1, I_2,\ldots,I_n\},\qquad \mathcal{T}=\{t_1, t_2,\ldots,t_n\}
\end{equation}
\textit{Association rule data mining} is an unsupervised learning discipline trying to reveal systematics in the database.

\subsection{Support of a pattern}
A pattern $X\subseteq\mathcal{I}$ is said to have a \textit{support} equal to the number of elements of $\mathcal{T}$ in which $X$ is a subset. Support may be specified absolutely or relatively.

\subsection{Association rules}
An \textit{association rule} takes the form $X\Rightarrow Y$, where $X$ and $Y$ are disjoint patterns.

\end{document}