\documentclass[12pt, a4paper]{article}
\usepackage{amsmath}
\usepackage{amsfonts}
\usepackage{amsthm}
\usepackage{mathtools}
\newtheorem{theorem}{Theorem}[section]
\newtheorem{definition}{Definition}[section]
\numberwithin{equation}{section}
\usepackage{pgfplots}
\pgfplotsset{width=10cm,compat=1.9}
\graphicspath{ {img/} }
\DeclareGraphicsExtensions{.png, .jpg}

\title{Machine learning strategies}
\author{Kristian Wichmann}

\begin{document}
\maketitle

\section{Introduction}
This note describes a number of strategies, tips and tricks for successfully navigating actual machine learning projects.

\subsection{The project cycle}
Basically, the evolution of a ML project over time comes in cycles of three phases:
\begin{equation}
\textrm{Idea}\rightarrow\textrm{Code}\rightarrow\textrm{Experiment}
\end{equation}
In short: Start with an idea, implement it in code. Then get your hands dirty and experiment with running the code in different ways. Eventually, this will hopefully lead to getting a new and hopefully better idea, which will start the cycle over again, until a satisfactory product has been build.

\subsection{Orthogonalization}
\textit{Orthogonalization} is essentially knowing what to tune to achieve what effect.

Ideally, when turning one "knob" on you ML problem, it should only affect one aspect of the outcome - each does only one, easily interpretable, thing. That way, you may tune one "knob" at a time to get the desired behaviour from the algorithm.

\subsubsection{Chain of assumptions}
When building an ML product, there's a chain of assumptions being made:
\begin{itemize}
\item We can fit the parameters well to the training set, so that the chosen cost function is low. In other words, we assume that we can get an acceptable performance on the training set. Sometimes this means getting human level performance or better.
\item We then hope this model also does well on the dev set. Failure to do so us typically a sign of overfitting.
\item And on the test set as well.
\item This should lead to good performance out in the real world.
\end{itemize}
For each of these problems, we have different "knobs" to turn in order to get the desired result. Here's some examples that may help achieve succes for each:
\begin{itemize}
\item Performance on the training set:
	\begin{itemize}
	\item Training a bigger network, or more generally a model with greater capacity.
	\item Choosing a better strategy for gradient descent. ADAM for instance
	\end{itemize}
\item Performance on the dev set:
	\begin{itemize}
	\item Regularization, like ridge regression or dropout.
	\item Getting a bigger training set.
	\end{itemize}
\item Performance on the test set:
	\begin{itemize}
	\item Get a bigger dev set.
	\end{itemize}
\item Performance in the real world:
	\begin{itemize}
	\item Change the dev set to better correspond to real world data.
	\item Change the cost function to better fit the problem.
	\end{itemize}
\end{itemize}
We'll dive into a lot of these in more detail below.

\subsubsection{Non-orthogonal example: Early stopping}
Some strategies are non-orthogonal. I.e. they affect several of the points in the chain above at the same time. One such example is early stopping, where performance is monitored on the training and dev set simultaneously. Therefore, in this case we're "tuning two knobs" at once.

\section{Setting your goal}
This is the crucial question for training the model: What are our optimization criteria? Which quantities do we want to optimize, and under which contraints (if any)? This section outlines a number of strategies, and thoughts on when/if it's a good idea to change these goals along the way.

\subsection{Single evaluation metric}
In this case, there's simply one quantity which gauges the performance of the model. We can then train the model to optimize the quantity. The cost function is an example.

If we have a number of discrete options to choose from, we can simply pick the one with the highest/lowest performance.

Often, there will be several relevant metrics related to a problem. Sometimes we can combine these into one, as the following example shows.

\subsubsection{Example: $F_1$ score}
For classification, both precision and recall are relevant metrics to consider. If we have trained two classifiers A and B with the properties when evaluated on the dev set shown in table \ref{table:two_classifiers}, it is not immediately clear which one we should pick; there's often a trade-off between the two.

\begin{table}
\centering
\label{table:two_classifiers}
\begin{tabular}{l|c|c|}
\cline{2-3}
                                   & \multicolumn{1}{l|}{Precision} & \multicolumn{1}{l|}{Recall} \\ \hline
\multicolumn{1}{|l|}{Classifier A} & 95\%                           & 90\%                        \\ \hline
\multicolumn{1}{|l|}{Classifier B} & 98\%                           & 85\%                        \\ \hline
\end{tabular}
\caption{Properties of two classifiers}
\end{table}

One way to combine the two, is the $F_1$ score, which is the harmonic mean of the precision ($P$) and recall ($R$):
\begin{equation}
F_1=\frac{2}{\frac{1}{P}+\frac{1}{R}}=\frac{2}{\frac{P+R}{PR}}=\frac{2PR}{P+R}
\end{equation}
We may now calculate the $F_1$ score of the two classifiers:
\begin{equation}
A: F_1=\frac{2\cdot 0.96\cdot 0.90}{0.96+0.90}\approx 92.4\%\quad
B: F_1=\frac{2\cdot 0.98\cdot 0.85}{0.98+0.85}\approx 91.0\%
\end{equation}
So, with $F_1$ score as the single evaluation metric, we should pick classifier A.

\subsection{Satificing and optimizing metrics}
Often, it is not possible to include all the relevant metrics into one. We can get around this problem by deciding on a metric that we want to optimize, and for the others decide on a performance that is good enough. The latter as known as satificing metrics.

\subsubsection{Example: Image classification}
We have trained three image classifiers, and their accuracies on the dev set as well as run time is shown in table \ref{table:satisficing}. We want to optimize accuracy with run time as a saticficing metric, so that is should be below 100 ms.

\begin{table}
\centering
\label{table:satisficing}
\begin{tabular}{l|c|c|}
\cline{2-3}
                                   & \multicolumn{1}{l|}{Accuracy} & \multicolumn{1}{l|}{Time/ms} \\ \hline
\multicolumn{1}{|l|}{Classifier A} & 90\%                           & 80                        \\ \hline
\multicolumn{1}{|l|}{Classifier B} & 92\%                           & 95                        \\ \hline
\multicolumn{1}{|l|}{Classifier C} & 95\%                           & 1500                        \\ \hline
\end{tabular}
\caption{Three image classifiers}
\end{table}

The satisficing condition disqualifies classifier C, even though it has the highest accuracy. Of the two that are left, classifier B is picked, as it has the highest accuracy.

\subsection{Distributions of dev vs. test sets}
In a sense, out goal is set by choosing the metric/loss function along with the dev set. Together, these can be thought of as a dart board target. Through the idea $\rightarrow$ code $\rightarrow$ experiment cycle, an ML project team can get progressively closer to hitting the bullseye of this dart board.

But when the time comes to checking the performance at the test set, we might be in trouble if we have not chosen it from the same distribution as the dev set! If these differ, we have actually "moved the target" for the test set, and hence the performance will suffer.

\subsubsection{Example: Geographical differences}
Let's say our data are divided into geographical regions: North America, South America, Europe, Asia, Africa, Australia. Now, imagine we choose our dev and test sets as follows:
\begin{itemize}
\item Dev set: North America, South America, Europe
\item Test set: Asia, Africa, Australia
\end{itemize}
Here, we would be in trouble, since these represents to "different targets" (two different distributions): Good performance on the dev set, may not correspond to good performance on the test set!

Instead, all of the data should be shuffled, and randomly distributed between the dev and test sets. Then the two sets come from the same distribution.

\subsubsection{Size of train, dev and test sets}
Earlier, it might have been adviced to have a split in size between the three sets be 60\% training set, 20\% dev set, and 20\% test set (or figures around these lines).

However, in the big data era, we often have a so much data, that there's no reason to make the dev and test percentages so large.

The test set should be large enough, that it ensures confidence in the performance of the model. The dev set should be of comparable size. Exact measures depend on the problem, but if, for instance we have a million data points, 1\% is 10000 samples! Often, this will be enough for confidence in the model. So, a reasonable split for such modern data sets is often closer to 98\% training set, 1\% dev set, and 1\% test set.

Sometimes, you may even forego having a test set, and simply work with a dev set (which confusingly, is then sometimes referred to as the test set). This is generally not recommendable, but may work nonetheless.

\subsection{When to change metrics and/or dev/test sets}
Sometimes, we find that the target we have set is simply not appropriate for what we wish to achieve. This section offers some insight into detecting such a problem, and how to deal with it.

\subsubsection{Example: Cat image classifier}


\section{Comparison to human-level performance}


\section{Case study: Bird recognition}


\section{Error analysis}


\section{Mismatched training/dev/test sets}


\section{Learning from multiple tasks}


\section{End-to-end deep learning}


\section{Case study: Image recognition for driverless car}

\end{document}